\chapter{Literature Review}
\label{Cap:Lit}
%

This chapter depicts the concept and characteristics of solar air heating collectors -- unglazed and glazed collectors. Considerations were given to:

\begin{itemize}
	\item Elements and factors to affect collector's performance;
	\item Optical concentrator, specifically ACPC with inverted absorber;
	\item Building integration of solar thermal systems.
\end{itemize}


\section{Solar air heating collectors}

Solar air heating collectors (SAHCs) are equipment designed to receive solar radiation and convert it into heat for working air heating. They are widely applied in many commercial applications, such as hot air supply to shopping malls, agricultural barns, industrial drying, etc. They are usually low-cost, with no freezing and high-pressure problems. Compared to water-heating solar collectors, SAHCs are outperformed due to their air thermal properties (\cite{Buker2015}). To overcome this challenge, the heat transfer from the hot absorber surface of the collector to the air needs to be enhanced while the collector's overall heat losses are minimised. (\cite{Shams2013}).

SAHCs can be classified into two main types: unglazed and glazed. The fundamental difference between these types is the presence of a glazing cover and the shape of the absorber surface. Unglazed air heating collectors (also known as Unglazed Transpired Solar Collectors -- UTSCs) do not have any glazing cover; they are composed of a perforated (transpired) absorber plate, as shown in Figure \ref{unglazed-glazed}(a). The absorber plate can be integrated into building fa{\c c}ades. The contact between the absorber plate and ambient air is increased by drawing air through the multiple perforations into the cavity (also called plenum) between the plate and the fa{\c c}ade (\cite{Shukla2012}). A fan pulls this heated air in the plenum into the building (\cite{Buker2015}).

The other main type is the glazed SAHC (or Glazed Air Heating Collector -- GAHC), which has at least a flat glazing cover to prevent the absorber from being exposed to the ambient and avoid efficiency losses. A typical design can be seen in Figure \ref{unglazed-glazed}(b). The air is pulled into the collector by a fan to be heated after contact with the absorber plate, which can have different shapes and/or other modifications to enhance the transfer mechanism.

\Figure[scale=0.58,placement=!ht,label={unglazed-glazed},caption={Illustration of (a) an unglazed and (b) a glazed solar air heating collector. From \citet{Kutscher1994} and \citet{SolarTribune2011}.}]{figs/unglazed-glazed.png}    

To improve the heat transfer from the hot absorber surface to the working air, a wide range of designs for SAHCs has been studied and reported in the literature: glazed, unglazed, bare plate, back-pass, perforated, un-perforated, single, double or triple passes, etc. (\cite{Kutscher1994}; \cite{Christensen1997}; \cite{Gawlik2005}; \cite{Koyuncu2006}; \cite{Leon2007}; \cite{Tchinda2008}; \cite{El-Sebaii2010}; \cite{Athienitis2011}; \cite{Zheng2016}; \cite{Li2016}). %To understand how SAHCs' thermal efficiency is improved, it is worthy defining the response variables to quantify the performance, depicted as follows: 

\subsection{Energy analysis}

Defining the response variables is worth quantifying to understand how SAHCs' thermal performance is improved. The thermal efficiency is defined as the ratio of useful energy rate collected by the working fluid to the incoming total solar radiation $\rm{I_{\!_T}}$ (W/m$^{2}$) received on the aperture area $\rm{A_{apt}}$ (m$^{2}$), calculated by Eq. (\ref{ThermalEf0}). Such efficiency can be evaluated either instantaneously or as an average over a certain period (\cite{Goswami2015}):

\begin{equation}
	\mathrm{\mathlarger{\eta}_{\mathrm{th}} = \frac{Q_u}{I_{\!_T}A_{apt}}}
	\label{ThermalEf0}
\end{equation}

\noindent where the useful energy rate $\rm{Q_u}$ (W) can be calculated considering the airflow rate $\rm{m_{air}}$ (kg/s), the temperature difference between outlet and inlet (\textdegree C), and the specific heat of the air (J/(kg.\textdegree C)) at constant pressure:

\begin{equation}
	\mathrm{{Q_u} = {m_{air}}{C_{p,air}}({T_{out}} - {T_{in}})}
	\label{usefulenergy0}
\end{equation}

The thermal characterisation of an SAHC relates the thermal efficiency under steady-state conditions to each temperature rise normalised by the corresponding solar radiation according to the Hottel-Whillier-Bliss equation, expressed by Eq. (\ref{hottel-whiller-eq0}).

\begin{equation}
	\mathrm{\eta_{\rm{th}} = \eta_o - {U_{\!_L}}\frac{(T_{abs} - T_{amb})}{I_{\!_T}}}
	\label{hottel-whiller-eq0}
\end{equation}

\noindent where $\rm{U_{\!_L}}$ is the collector's overall heat loss coefficient (W/(m$^2$.\textdegree C)) and both absorber and ambient temperatures ($\rm{T_{abs}}$ and $\rm{T_{amb}}$) are in \textdegree C. The $\rm{U_{\!_L}}$ value depends weakly on temperature, and, in most cases, it is considered constant at typical operating conditions (\cite{Rabl1985}). Lastly, the optical efficiency $\eta_{\rm{o}}$ is the ratio between the absorbed and the incident solar radiation. From this equation, $\eta_{\rm{th}}$ can be plotted against $\rm{(T_{abs} - T_{amb})/I_{\!_T}}$, resulting in a linear curve, with $\eta_{\rm{o}}$ and $\rm{{U_{\!_L}}}$ as the linear coefficient and the slope, respectively (\cite{Goswami2015}). 

Another way to calculate the efficiency is based on the air temperature, particularly $\rm{T_{in}}$ and $\rm{T_{out}}$ at the inlet and outlet of the collector, respectively (\cite{Duffie2013}). Eq. (\ref{hottel-whiller-eq0}) can be rewritten by adding a multiplicative term known as the heat removal factor or heat exchange effectiveness. It is defined as the ratio of the heat transferred to the airflow to the maximum possible heat transfer if the outlet air temperature was heated to the absorber surface temperature (\cite{Kutscher1994}). In practice, as the air temperature approaches the absorber temperature, the heat transfer rate slows, requiring infinitely large surface areas or time to achieve the maximum possible heat transfer. This effectiveness is calculated as follows:

\begin{equation}
	\mathrm{\varepsilon_{\!_{HX}} = {\frac{{{T_{out}} - {T_{in}}}}{{{T_{abs}} - {T_{in}}}}}}
	\label{heat-exchange}
\end{equation}

This effectiveness can also be written as a function of the airflow rate and the convective heat transfer coefficient ($\rm{h_{\!_{HX}}}$):

\begin{equation}
	\mathrm{\varepsilon_{\!_{HX}} = 1 - exp\left(-\frac{h_{\!_{HX}}A_{abs}}{m_{air}C_{p,air}}\right)}
	\label{effectiveness}
\end{equation}

\noindent where this convective coefficient $\rm{h_{\!_{HX}}}$ (W/(m$^2$.\textdegree C)) considers the heat transfer mechanism between the airflow and the hot absorber plate. It is related to the Nusselt number, which is the ratio of convective to conductive heat transfer (\cite{Incropera2006}). This is calculated by Eq. (\ref{Nusselt}):
	
\begin{equation}
	\mathrm{Nu = \frac{L_c}{k_{air}}{h_{\!_{HX}}}}
	\label{Nusselt}
\end{equation}

\noindent where $\rm{L_c}$ is the characteristic length of the heat transfer (m) and $\rm{k_{air}}$ is the air thermal conductivity (W/(m.\textdegree C)). The Nusselt number is usually a function of the Reynolds number, which is proportional to the air velocity. Therefore, operating a solar collector with higher airflow rates is desirable to enhance heat transfer. In addition to that, many other factors influence the heat transfer mechanism. The following section describes the elements of a SAHC and how they influence its thermal performance. These elements are the absorber plate, glazing cover, air flow rate and system modifications.

\subsection{Effect of absorber surface}  

As the airflow to be heated goes through a perforated absorber surface, the heat transfer mechanism between the holes and the air is more effective compared to flat plate collectors. The absorber design considers the following factors: i) material and coatings, ii) shape, iii) thickness of the absorber, and iv) modified area.

\subsubsection{Absorber material and coatings} 

It was previously assumed that a collector's high performance resulted from highly thermally conductive materials. This assumption was based on the fact that a significant portion of the thermal energy was transferred to the airflow. Plus, it would considerably increase the average surface temperature and, consequently, the radiative heat loss to the ambient (\cite{Gawlik2002}; \cite{Gawlik2005}).

\citet{Christensen1997} compared experimentally the thermal performance of an UTSC for two absorber materials: aluminium (k = 216 W/(m.K)) and styrene plastic (k~=~0.16~W/(m .K)), where both surfaces had the same geometry. Results show that the UTSC's efficiency is relatively insensitive to low thermal conductivity materials. This means sufficient thermal energy is from the absorber surface to the air, so high thermal conductivity materials are not as critical as previously assumed.

\citet{Arulanandam1999} simulated a UTSC's thermal performance in CFD under no wind conditions, varying the absorber's thermal conductivity from 0.196 to 15.121~W/(m.K)). They concluded that, for low porosity plates, the heat exchange effectiveness dropped by 10 -- 20\%, but the thermal efficiency only decreased by approximately 5\%.

The temperature distribution along the absorber surface can vary for low-conductive materials. The local convective heat transfer is higher in regions of higher surface temperature and lower in regions of lower surface temperature. This effect is reduced if the hole pitch is small so that a large temperature gradient is not achieved. \citet{Gawlik2005} showed that the effect of material conductivity on thermal performance is negligible. The study then concluded that low thermal conductivity materials can be used with no thermal efficiency penalty but a significant benefit in cost savings and corrosion resistance.

One or both of the front and back surfaces of the absorber may be dark in colour to absorb solar radiation. Surfaces may be treated or coated to have a low emittance of infrared radiation to reduce heat losses. The effect of the radiative properties absorptivity and emissivity has also been reported. \citet{Leon2007} simulated a UTSC varying the values of both properties and found that absorptivity substantially affects thermal efficiency more than emissivity. \citet{El-Sebaii2010} simulated a single-pass GAHC comparing absorber surfaces using different coatings. They concluded that the highest daily efficiency was achieved using nickel–tin (absorptivity and emissivity of 0.98 and 0.14). Compared to a black-painted galvanized iron absorber (0.88 of absorptivity and emissivity), the nickel-tin coated absorber outperformed by 29.23\%. \citet{Li2016} simulated a glazed TSC and found that painting the absorber plate with a selective coating of higher absorptivity and lower emissivity would enhance the heat transfer mechanism from the absorber plate to the plenum and reduce the radiative losses from the absorber surface to the glazing cover. The effect of coating absorptivity on thermal efficiency is more considerable than emissivity's.

\subsubsection{Effect of porosity, perforation diameter and pitch} 

The plate porosity for flat plates is defined as the ratio of perforation area to the total surface area. It can only be calculated by setting values of perforation diameter and pitch. Smaller perforation diameters strengthen the jet impingement and thus increase the heat transfer mechanism (\citet{Li2016}). The pitch (or perforation spacing) more substantially influences heat exchange effectiveness than thermal efficiency. With larger pitches, hot spots tend to develop on regions of the absorber surface away from the perforations (\citet{Arulanandam1999}). However, this effect can be avoided if the distance between the holes is small enough.

\citet{Kutscher1994} investigated the effect of increasing perforation diameter (1.6 mm to 3.2 mm) and pitch (13.5 mm to 27 mm) and verified that the heat exchange effectiveness decreased. \citet{Arulanandam1999} discovered that the Nusselt number was increased in a range of plate porosity from 0.5\% to 2\%. \citet{Decker2001} simulated an UTSC and found that the heat exchange effectiveness decreased with increasing pitch (7 mm to 24 mm) and perforation diameter (0.8 mm to 3.6 mm). They also concluded that 28\% of the air temperature rise occurs in holes of the perforated plate. \citet{Gawlik2005} tested two plates of different porosities (0.3\% and 5\%) and concluded that there was no significant difference in air temperature rise because the perforation diameter was increased and the pitch was decreased by the same proportion. 

According to \citet{Leon2007}, simulation results showed that the airflow temperature rise and heat exchange effectiveness increased with decreasing pitch and perforation diameter. For a constant solar radiation and airflow rate, changing the pitch from 12 to 24 mm with a corresponding change in perforation diameter from 0.8 mm to 1.55 mm resulted in a drop of 5.5 $^{\rm o}$C in the airflow temperature rise. Furthermore, when the porosity was increased, the effectiveness and the thermal efficiency decreased.

\subsubsection{Effect of absorber thickness}   

The effect of the absorber thickness on thermal effectiveness is insignificant when it is smaller than 1.5 mm (\cite{Kutscher1994}). However, for thick absorbers (between 1.6 and 3.6 mm), the heat transfer varies between different perforation areas, thus affecting the collector's thermal performance. \citet{Zomorodian2012} tested UTSCs with two different absorber plate thicknesses and concluded that the thicker was more efficient.

\subsubsection{Effect of the absorber-modified area}

One way to enhance the heat transfer from the absorber to the airflow is to introduce obstacles in the air stream to increase the absorber surface area. These obstacles can be fixed either on the internal face of the absorber, on the back plate or as a combination. These obstacles aim to increase the outlet air temperature and efficiency with minimum losses (\cite{Karsli2007}). Modifications of the absorber area include using finned, wavy, or V-corrugated shapes, as seen in Figure \ref{modified}.

\Figure[scale=0.62,placement=!ht,label={modified},caption={Illustration of GAHCs with different absorber areas: (a) wired, (b) wavy, (c) finned, and (d) V-shaped (or V-corrugated). From \citet{Pottler1999}.}]{figs/modified_area.png}

\citet{Karim2004} studied and compared three types of GAHCs: flat plate, finned and V-corrugated to achieve an efficient design suitable for a solar dryer. They found that the V-corrugated collector is the most efficient and the flat plate one the least. Results show that the V-corrugated collector has 7 -- 12\% higher efficiency than flat plate collectors. \citet{Kurtbas2004} also studied different absorber geometries and concluded that all GAHCs outperform the flat plate collector. \citet{Karsli2007} compared the thermal performance of a finned absorber collector to a flat plate one with no fins and concluded that the finned GAHC is more efficient than the flat plate. This is because the fins create turbulent flow, which leads to a higher heat transfer coefficient, lowering the absorber temperature and reducing the thermal heat loss at the same time. \citet{Alta2010} compared a GAHC with a flat absorber surface to one with a finned absorber and concluded that attaching fins on that surface increases the thermal efficiency. \citet{Assari2011} developed a mathematical model based on the effectiveness method for assessing the thermal performance of a GAHC, where water and air flow simultaneously. Three different channels were used to enhance the collector's performance: rectangular fin, triangular fin (V-corruated) and without fin. Simulation results show that channels with rectangular fins performed the best. \citet{ElSebaii2011} compared the thermal performance of a flat plate collector to a V-corrugated collector. The results showed that the double-pass V-corrugated plate collector is 11 -- 14\% more efficient than the double-pass flat plate collector.

\subsection{Effect of glazing cover} 

A glazing cover plays an important role in suppressing convective heat losses from the absorber plate to the ambient and protecting the solar collector against weather conditions. It also needs to have high transmissivity to the solar spectrum and be extensively opaque to long (or infrared) wavelength radiation emitted by the absorber (\cite{Saxena2015a}). In other words, a glazing cover reduces convective and radiative heat losses while transmitting most incoming solar radiation (\cite{Norton2006}).

Glazing materials commonly used are glass, plastic, and fibreglass. The challenge is to find a material with high transmissivity and low thermal conductivity that is affordable and has the required mechanical properties for building integrated systems. Glass is considered a glazing cover because it is transparent in the solar range and absorbs almost all the infrared radiation re-emitted by the absorber plate. This enhances the collector's thermal efficiency by creating a greenhouse effect (\cite{Khoukhi2006}).

More than one glazing cover may be used to minimise the heat losses from the collector (\cite{El-Sebaii2010}; \cite{Yeh2009}). However, the transmissivity decreases due to increased reflections (\cite{Michalopoulos1994}). \citet{Alta2010} conducted an experimental study comparing a single and double-glazed flat plate collector with a finned absorber. The authors stated that the double-glazed collector performed better and showed a higher air temperature difference.

Reflection losses at the glazing surface depend on the refractive index of the cover material and the structural orientation of the glazing. The lowest reflection losses can be achieved if an antireflective coating with a refractive index between the air and the cover material is used. Investigations have shown that the glass transmittance and solar collector efficiency can be increased by 4\% if an antireflection coating is applied instead of standard glass as the cover plate for the solar collector (\cite{Furbo2003}).  

Glass is also very resistant to scratching and high operating temperatures and practically impervious to the damaging effects of ultraviolet exposure. The incident sunlight transmitted through glass depends on the iron content of the glass, which varies between 85\% and 92\% at normal incidence. While low iron content increases the transmittance of glass, low iron glass is more expensive. Glass can be easily broken, but this can be minimized by using tempered glass, which adds a further cost (\cite{Duffie2013}).

Transparent plastics, such as polycarbonates, polyethylene and acrylics have also been used as glazing materials (\cite{Koyuncu2006}). Their main advantages are resistance to breakage and light weight, and are cheaper than glasses. Their main advantages are resistance to breakage, they are lightweight, and they are cheaper than glasses. The main disadvantages of plastics are high transmittance in the longer wavelength, and deterioration over time due to ultraviolet solar radiation (\cite{Goswami2015}; \cite{Duffie2013}). Additionally, plastics are generally limited in the temperatures they can sustain without deteriorating or undergoing dimensional changes. Although glass is expensive, this is the best glazing material considering 20 years of collector life span (\cite{Shams2013}).

\subsection{Effect of airflow rate}  

The flow rate is the most important factor in the thermal performance of a solar collector. The effects of that have been extensively studied. Experimental and simulation results show that more useful heat is collected when the flow rate increases and overall losses are reduced, thus increasing the collector's thermal efficiency. On the other side, the outlet temperature and the heat exchange effectiveness are decreased (\cite{Christensen1997}; \cite{Ammari2003}; \cite{Leon2007}; \cite{El-Sebaii2010}; \cite{Zomorodian2012}; \cite{Li2016}). A typical relationship between air temperature rise, thermal efficiency and flow rate is shown in Figure \ref{airflow_effect}.

\Figure[scale=0.70,placement=!ht,label={airflow_effect},caption={Graphs of air temperature rise and thermal efficiency against mass flow rate at the steady state. From \citet{Tyagi2012}.}]{figs/airflow_effect.png}

\citet{Ammari2003} simulated a single-pass GAHC and found that the thermal efficiency increased until a certain level of volumetric flow rate, and then it increased at a lower rate. A decreasing pattern could be observed for the air temperature as the flow rate was increased. \citet{Leon2007} simulated a UTSC and observed that the decrease in energy rate is lower at lower approach velocities. They also concluded that the collector efficiency decreases with increasing outlet air temperatures. \citet{Jafarkazemi2013} simulated a flat plate collector and also observed that the overall heat loss coefficient dropped as the mass flow rate increased. \citet{Badache2014} and \citet{Nowzari2015} analysed different factors affecting UTSC and GAHC performances and concluded that the airflow rate strongly influenced thermal efficiency. \citet{Li2016} evaluated the performance of a GAHC with a perforated absorber, considering the fan power required to overcome airflow resistance through the collector. In this case, higher airflow rates lead to higher useful heat collected and fan power required. Results show that there is a level of airflow rate that maximises energy efficiency.

\subsection{System modification}

This topic depicts the influence of the airflow passing through the collector in contact with the absorber, the parameterisation of the channel dimension through which the air flows, and how multiple collectors are connected (series or parallel).

\subsubsection{Air passage through the collector}

Considerations are given to the airflow contact with the absorber plate (or PV module) when the former passes through the collector. A single-pass collector is when the airflow comes into contact with the absorber only once. Alternatively, it is double-pass collector when air flows in contact with the absorber twice, as shown in Figure \ref{double_pass}. 

\Figure[scale=0.85,placement=!ht,label={double_pass},caption={Illustration of (a) single pass and (b) double pass air heating collector. From \citet{Hegazy2000}.}]{figs/double_pass.png}

\citet{Hegazy2000} simulated a photovoltaic/thermal (PV/T) glazed collector using different modes: air flowing over or below the absorber and even on both sides in a single or in a double pass. The author found out that flowing air on both sides leads to the best performance, considering both applications. \citet{Yousef2008} tested single and double-pass GAHCs and concluded that double-pass achieved higher thermal efficiency and outlet air temperature. \citet{Nowzari2015} investigated a single and double-pass GAHC and concluded that higher efficiencies are achieved using the double-pass mode. \citet{Karim2004} investigated collectors with different absorber configurations. From the efficiency tests in double-pass operation, it is concluded that the efficiency of all collectors increases in double-pass mode.

The effect of external recycling on the GHAC thermal efficiency has been investigated. If operated with an external recycle, the collector's thermal performance is considerably improved, and the desirable effect overcomes the drawbacks. The performance is enhanced with increasing reflux ratio, especially for operating at a lower air flow rate with higher inlet air temperature (\cite{Yeh2009}).

\subsubsection{Channel's dimensions}

\citet{Hegazy1999} showed a criterion for determining the channel's optimum depth-to-length ratio, which maximises the useful energy from GAHCs to operate at a fixed mass rate of airflow. The author also observed that decreasing the depth or increasing the length improves the thermal performance. \citet{Tonui2007} studied a photovoltaic/thermal glazed collector where forced air was used to extract heat from the back of the PV module through a channel. They evaluated the effect of the channel depth (distance between the PV module and the collector's bottom) and the length on the thermal performance at a constant airflow rate. It was verified that the thermal efficiency and air outlet temperature were reduced with increasing channel depth. They also concluded that the thermal efficiency increased with increasing channel length and approached a constant value as the length increased. The same pattern was verified by \citet{Yousef2008} when they investigated the thermal performance of a single-pass GAHC where forced air was pumped above the absorber plate. They also concluded that air temperature and efficiency decreased with increasing channel depth at a constant airflow rate. \citet{Tonui2008} simulated a photovoltaic/thermal glazed collector using natural convection to alleviate the PV module temperature. They found that there is an optimum channel depth that maximises thermal efficiency.

\subsubsection{Connection of collectors}

Solar collector arrays can be connected in series or parallel, each with different thermal performance outcomes. In a series connection system, the total heat transfer of the collectors is higher since the total airflow passes through all collectors. However, the temperature rise in the collectors reduces the thermal efficiency of each progressive collector, and the power consumption to pump the airflow is also higher (\cite{Hastings2000}). A schematic of multiple units in connection is shown in Figure \ref{series-parallel}. 

\Figure[scale=0.65,placement=!ht,label={series-parallel},caption={Schematics of a system with collectors in series and parallel. From \citet{Fanney1981}.}]{figs/series-parallel.png}

\citet{Oonk1979} developed a formula to predict the performance of N collectors in series by deriving the exit temperature of one collector and using it as the inlet temperature of the next collector. \citet{Fanney1981} also developed an equation to predict the thermal performance of collector arrays, both in series and parallel, based on energy balance.

\section{Concentrating collectors}

When it is required to warm the airflow to higher temperatures with high thermal performance, concentrators can be employed to meet the objective. The working principle is to receive solar radiation through a larger area and direct that, using reflectors, to a smaller absorber area. There are different types of concentrating collectors, depicted as follows (\cite{Evangelisti2019}): 

\begin{itemize}
	\item Parabolic trough collector (PTC): consists of a parabolic reflector and an absorber placed along the whole length of the concentrator at the parabola's focus. PTCs generally track the Sun using east-west, north-south, or polar orientations. The absorber is usually tubular and enclosed in a glass tube to reduce radiative and convective losses (\cite{Goswami2015}). The solar radiation is reflected torward the tube, bringing it to high temperatures and heating the heat-transfer fluid that flows inside it. The absorber temperature varies between 50 $^{\rm{o}}$C and 300 $^{\rm{o}}$C, but it can also reach \mbox{400 $^{\rm{o}}$C};
	
	\item Compound parabolic concentrator (CPC): A CPC consists of two distinct parabolic segments, with each segment's focus positioned at the opposite end of the absorber surface. The axes of these parabolic segments are tilted away from the CPC's central axis by the acceptance angle $\theta_{{\rm{acc}}}$;
	
	%It is designed from two distinct parabolic segments, where the focus of each one is located at the opposing absorber surface end points. The axes of the parabolic segments are oriented away from the CPC axis by the acceptance angle $\theta_{{\rm{a}}}$. The slope of the reflector surfaces at the aperture is parallel to the optical axis for untruncated CPCs;
	
	\item Heliostat field collector: composed of flat reflectors placed all around a central receiver, called solar tower. These reflectors can face the Figure through a tracking system. Central receivers can achieve temperatures of 1000 $^{\rm{o}}$C or even higher. Therefore, a heliostat collector is suitable for thermal electric power production of 10 -- 1000~MW (\cite{Goswami2015});
	
	\item Linear Fresnel collector: comprised of linear receivers and reflectors. The reflector segments are aligned horizontally, facing the Sun so that the rays can hit the receiver without needing movement;
	
	\item Parabolic dish collector: characterized by a paraboloidal geometry that concentrates solar radiation onto a receiver placed at the focal point of the collector.
	
\end{itemize}

\subsection{Optical analysis}

One way of evaluating a concentrator is by characterising its optical performance. Optical efficiency is defined as the ratio between the absorbed and the incident solar radiation. For this analysis, the optical properties of the reflectors, glazing cover and absorber should be taken into consideration, as well as the reflectors' shape and the concentration ratio (\cite{Sellami2013}):

\begin{equation}
	\mathrm{\eta_o = {\tau_{col}}\tau_{glaz}\alpha_{abs}}
	\label{optical0}
\end{equation}

\noindent where the term $\rm{\tau_{col}}$ considers the number of reflections at the reflectors. The reflector's shape and concentration ratio dictate the number of reflections and, consequently, the optical efficiency. The term $\rm{\tau_{glaz}}$ is the transmissivity of the glazing cover; alternatively, for collectors without cover or glass-tube absorbers, this term is unity. The term $\rm{\alpha_{abs}}$ is the absorptivity of the absorber plate. Therefore, it is desirable that the concentrator transmits most of the solar radiation through the glazed aperture, has a highly reflective reflector surface, and has a high absorptivity absorber with low emissivity. Most of these properties have been discussed in the previous section of this Chapter.

Other factors are taken into consideration: i) incoming solar radiation and its components; ii) the position of the Sun at a specific location and what direction the concentrator is facing at a set inclination; iii) the optics in the glazing cover, and; iv) the effect of truncation. Other factors, such as the parabolic shape of a CPC collector and concentration ratio, will be further investigated in another section.

%The shape of reflectors affects size, angular acceptance range (\cite{Zacharopoulos2000}; \cite{Harmim2012}) and maximum geometric concentration ratio (\cite{Mills1978}). Given the same level of concentration ratio, concentrators of narrow angular acceptance are more efficient so collect higher amounts of the available solar energy (\cite{Sarmah2011}; \cite{Kostic2012}).

%A concentrator is usually truncated. Less reflector material is then required which reduces costs and weight. Figure \ref{untruncated} shows the untruncated and truncated versions of the same concentrator. The effect of truncation on the design parameters concentration ratio, height-aperture width ratio and reflector length has been presented by graphs and equations (\cite{McIntire1979}; \cite{Rabl1976}). Figure \ref{ang_acc} shows the angular acceptance function for an untruncated (full) CPC and for a truncated configuration.
%
%\Figure[scale=0.40,placement=!ht,label={untruncated},caption={Truncated and untruncated CPC with a tubular absorber.}]{figs/untruncated.png}
%
%\Figure[scale=0.60,placement=!ht,label={ang_acc},caption={Angular acceptance function of a full and truncated CPC. Adapted from \citet{Norton1991}.}]{figs/ang_acc2.jpg}
%
%\citet{Carvalho1985} derived analytic expressions of angular acceptance range and evaluated the yearly collectable energy as function of the extent of truncation. Truncated concentrators accept solar rays at broader incident angles, thus collecting more solar energy due to reduction of reflections. \citet{Francesconi2018} simulated a five-CPC assembly's performance in CFD; they concluded that a concentration ratio reduction from 2.0 to 1.96 resulted in a 2\% increase of the system's thermal efficiency.

\subsubsection{Solar radiation and intercept factor}

The total solar radiation ($\rm{I_{\!_{T}}}$) incident on the concentrator's aperture is the sum of the beam ($\rm{I_{\!_{B}}}$) and diffuse ($\rm{I_{\!_{D}}}$) components (W/m$^{2}$). However, since concentrators exploit only part of the diffuse radiation, which is dependent of the collector's concentration ratio CR, the factor $\Gamma$ must be defined as the fraction of total solar radiation accepted. Assuming that the angular distribution of diffuse radiation is isotropic, this factor can be estimated as (\cite{Rabl1980}):

\begin{equation}
	\mathrm{\Gamma  = \frac{\displaystyle {\left( {{I_{\!_B}} + \frac{{{I_{\!_D}}}}{CR}} \right)}}{{{I_{\!_T}}}}}
	\label{xi}
\end{equation} 

\noindent and therefore the incoming solar radiation available to reach the absorber surface is assumed to be $\rm{I_{\!_T}\Gamma}$. Since CPC collectors operate in the concentration ratio range of 2 to 10 to capitalise on the corresponding reduced tracking requirement, diffuse radiation is accepted by one-half to one-tenth of the incident (\cite{Goswami2015}). 

It is important to highlight other angular distributions of the diffuse insolation that can affect the performance of PTC and CPC collectors. Two other particular distributions can be considered: cosine and hybrid Gaussian. The hybrid Gaussian distribution combines an isotropic background with a circumsolar Gaussian part. This distribution is more realistic for a tracking system than the isotropic model (which underestimates the insolation intensities at incidence angles near zero) and the cosine model (which underestimates the intensities at large incidence angles). The analytical expressions for the three distributions considered have been presented by \citet{Prapas1987}. However, on clear days (diffuse radiation is approximately 11\% of total radiation), the difference in optical efficiency between the three distributions of diffuse radiation is insignificant.

\subsubsection{Collector's inclination and orientation}

It is important to establish the position of the collector's aperture considering its inclination concerning the horizontal plane and the Sun's position as a function of the day, time and location. Figure \ref{Figure_orientation} shows a basic scheme of the Sun's position by the altitude ($\rm{a_s}$) and azimuth ($\rm{\gamma_s}$) solar angles, as well as the inclination $\beta$ and the collector azimuth angle $\rm{\gamma_{col}}$. The latter defines the orientation of the collector: if $\rm{\gamma_{col}} = 0$, the collector is south-facing.

\Figure[scale=0.80,placement=!ht,label={Figure_orientation},caption={(a) Altitude solar angle, surface azimuth angle, and solar azimuth angle for an inclined surface. (b) Plan view showing solar azimuth angle. Adapted from \citet{Duffie2013}.}]{figs/Figure_orientation.png}

The incident angle $\theta_{\rm{i}}$, defined as the angle between the incident solar ray and the normal to the aperture, is calculated by Eq. (\ref{incidence0}):

\begin{equation}
	\mathrm{\theta_i = {\cos ^{-1}}\left[ {\cos {a_s}\cos ({\gamma_s} - {\gamma_{col}})\sin \beta  + \sin {a_s}\cos \beta} \right]}
	\label{incidence0}
\end{equation}

Solar hour angles ($\omega_{\rm{s}}$) were obtained from the solar time for each day of operation. The solar altitude and solar azimuth angles were calculated from (\cite{Duffie2013}):

\begin{equation}
	\mathrm{{a_s} = {\sin ^{ - 1}}\left( {\sin \phi \sin {\delta _s} + \cos \phi \cos {\delta _s}\cos {\omega_s}} \right)}
	\label{solar_alt0}
\end{equation}
\vspace*{-0.5cm}
\begin{equation}
	\mathrm{\gamma_s = {\sin^{-1}}\left(\frac{\cos \omega_s \sin \delta_s}{\cos a_s}\right)}
	\label{azimuth0}
\end{equation}

\noindent where the solar declination $\delta_{\rm{s}}$ is a function of the year's day and $\phi$ is the location's latitude.

\citet{Pottler1999} calculated the total energy collected for a GAHC at north, south, west and east orientations located in the Northern Hemisphere and verified that the south-facing orientation collects more energy because it is optically more efficient. \citet{Roux2016} used a ray tracing simulator to calculate a flat plate collector's optimum inclination and azimuth angle at different locations. Results showed that an optimally positioned collector can, on average, collect 10\% more annual solar energy than a horizontally fixed collector. The optimum fixed inclination angle is similar to the location's latitude, and the optimum fixed azimuth angle is a function of the longitude angle minus the absolute latitude angle.

\subsubsection{Glazing optics}

By Snell's law, considering the angle of incidence at the glazing cover ($\theta_{\rm{i}}$), the angle of refraction is calculated by Eq. (\ref{snell}):

\begin{equation}
	\mathrm{\theta_r = arcsin\left(\frac{sin\theta_{i}}{n_{idx}} \right)  }
	\label{snell}
\end{equation}

\noindent where $\rm{n_{idx}}$ is the refraction index of the glazing cover dependent of the material. When passing through the glazing medium, part of the incident radiation is reflected and absorbed. The term that takes into account the reflected radiation alone is given by Eq. (\ref{rho_glaz}):

%\begin{equation}
%	\mathrm{r_{par} = \frac{tan^2(\theta_r - \theta_i) }{tan^2(\theta_r + \theta_i) } }
%	\label{parallel}
%\end{equation}
%
%\begin{equation}
%	\mathrm{r_{perp} = \frac{sin^2(\theta_r - \theta_i) }{sin^2(\theta_r + \theta_i) } }
%	\label{perpend}
%\end{equation}

\begin{equation}
	\mathrm{\rho_{glaz} = \frac{1}{2}\left[\frac{1 - r_{par}}{1 + (2N_c - 1)r_{par}} + \frac{1 - r_{perp}}{1 + (2N_c - 1)r_{perp}}\right]}
	\label{rho_glaz}
\end{equation}

\noindent where $\rm{N_c}$ is the number of glazing covers, and $\rm{r_{par}}$ and $\rm{r_{perp}}$ are the unpolarized components of the radiation, functions of the angles of incidence and refraction (\cite{Duffie2013}). Similarly, the term that considers the glazing absorptivity $\rm{\alpha_{glaz}}$ is given by Eq. (\ref{abs_glaz}):

\begin{equation}
	\mathrm{\alpha_{glaz} = exp\left(-\frac{K_{ext}\delta_{glaz}}{cos\theta_r} \right) }
	\label{abs_glaz}
\end{equation}

\noindent where $\rm{\delta_{glaz}}$ is the glazing thickness and $\rm{K_{ext}}$ is the extinction coefficient which is a function of the material: for glass, the value of this coefficient varies from approximately 4 m$^{-1}$ for ''water white'' glass to nearly 32 m$^{-1}$ for high iron oxide content (greenish cast of edge) glass. Lastly, the glazing transmissivity $\rm{\tau_{glaz}}$ can be approximated by the product of the glazing absorptivity and the standalone transmissivity as a function of the glazing material and the incidence angle:

\begin{equation}
	\mathrm{\tau_{glaz} \cong \rho_{glaz}\alpha_{glaz}}
	\label{transmi}
\end{equation}



\subsubsection{End losses}

If a linear concentrator is long in an east-west orientation compared to its width and height, it can be assumed to behave as a two-dimensional system where end effects are negligible (\cite{Eames1993a}). The optical analysis must include the end losses for concentrators with short axial lengths, as some reflected solar rays might not reach the absorber under certain obtuse solar incident angles. To account for the end losses, the optical efficiency is multiplied by a factor for that purpose, which is a function of the incidence angle and the collector's geometry. This factor was analytically calculated for imaging parabolic troughs (\cite{Rabl1985}) and linear Fresnel concentrators. End losses have a more substantial influence at high-latitude locations, which could cause the absorber to be entirely in shadow on winter days (\cite{Hongn2015}). \citet{Pu2011} estimated the end-effect for the north-south and east-west linear Fresnel collectors, analyzing the angles between incident solar rays and the tacking axes of the reflectors. They concluded that the end losses can be compensated by increasing the length of the mirror field. \citet{Heimsath2014} quantified optical losses of Fresnel collectors and proposed a corrective end loss factor for end loss modelling. They also observed that the optical losses are higher for wider zenith angles and shorter collectors. \citet{Xu2014} presented an optical analysis and compensation method for the end loss effect of parabolic trough solar collectors with a horizontal north-south axis. The calculation formulae for optical end loss ratio and increased optical efficiency are derived, and various factors affecting them are analyzed. The compensation method is found to be applicable for regions with latitudes over 25$^{\rm{o}}$ and short trough collectors.

\subsubsection{Truncation}

Compared to a simple parabola, a CPC is very deep, and therein lies its main disadvantage: it requires a relatively large reflector area for a given aperture area. For example, for a concentration ratio of 10, the ratio of reflector area to aperture area is approximately 11, while a PTC has a ratio of around 1.2. To overcome this drawback, a large portion of the top of a CPC can be cut off with almost no loss in performance. A CPC will almost always be truncated in practical applications for economic reasons (\cite{Rabl1976}). \citet{Carvalho1985} derived analytic expressions of the angular acceptance range and evaluated the yearly collectable energy as a function of the extent of truncation. Truncated concentrators accept solar rays at broader incident angles, thus collecting more solar energy due to reduced reflections. \citet{Francesconi2018} simulated a five-CPC assembly's performance in CFD; they concluded that a concentration ratio reduction from 2.0 to 1.96 resulted in a 2\% increase in the system's thermal efficiency.

\subsection{The Compound Parabolic Concentrator (CPC) family}

This section depicts the geometry of a symmetric CPC and an asymmetric CPC, as well as a modification to include an inverted absorber.

\subsubsection{Symmetric CPC} 

The compound parabolic concentrator (CPC) is a non-imaging solar concentrator that has the advantages (compared to focusing ones) of i) no need for solar tracking and ii) the ability to collect a portion of diffuse radiation (\cite{Winston1974}). The CPC has been used for heating and PV electricity production (\cite{Jaaz2017}). A typical CPC cross-section is shown in Figure \ref{CPC1}. Solar radiation accepted within the angular acceptance range is focused onto an absorber by reflection at the two symmetrical parabolic reflectors. 

\Figure[scale=0.60,placement=!ht,label={CPC1},caption={Basic CPC with flat plate absorber: (a) cross-section design in 2D and (b) in 3D. From \citet{Duffie2013} and \citet{Winston1974}.}]{figs/CPC-2D-3D.png}

One of the geometric parameters of a CPC is the geometrical concentration ratio, which is the ratio of the aperture to the absorber areas. This parameter has an upper limit calculated by Eq. (\ref{CR}):

\begin{equation}
	\mathrm{CR = \frac{1}{sin\theta_{acc}}}
	\label{CR}
\end{equation}

\noindent where the half-acceptance angle $\rm{\theta_{acc}}$ is the angular limit over which radiation is fully accepted without moving all or part of the concentrator (\cite{Rabl1976a}). If the reflectors are specular, all the solar rays incident at angles within the acceptance range (between $\pm \theta_{\rm{acc}}$) reach the absorber surface. Figure \ref{untruncated} shows the untruncated and truncated versions of the same CPC. The effect of truncation on the design parameters concentration ratio, height-aperture width ratio and reflector length has been presented by graphs and equations (\cite{McIntire1979}; \cite{Rabl1976}). Figure \ref{ang_acc} shows the angular acceptance function for an untruncated (full) CPC and a truncated configuration.

\Figure[scale=0.40,placement=!ht,label={untruncated},caption={Truncated and untruncated CPC with tubular absorber. From \citet{Norton1991}}]{figs/untruncated.png}

\Figure[scale=0.60,placement=!ht,label={ang_acc},caption={Angular acceptance function of a full and truncated CPC. Adapted from \citet{Norton1991}.}]{figs/ang_acc2.jpg}

The shape of reflectors affects the size, angular acceptance range (\cite{Zacharopoulos2000}; \cite{Harmim2012}) and the maximum geometric concentration ratio (\cite{Mills1978}). Given the same level of concentration ratio, concentrators of narrow angular acceptance are more efficient, so they collect higher amounts of the available solar energy (\cite{Sarmah2011}; \cite{Kostic2012}).

The CPC has been extensively studied and reported in the literature. A common case of study considered a CPC where a working fluid passes within the absorber to capture the heat. However, the cavity between the absorber and the glazing cover is filled with dead air, and therefore, convective losses occur due to the temperature gradient. This affects the motion of air, known as natural or free convection. Hence, this free convective heat transfer coefficient $\rm{h_f}$ (W/m$^2$) is calculated as:

\begin{equation}
	\mathrm{h_f = \frac{{{k_{air}}}}{{{L_c}}}Nu_{f} = a{(Ra)^b}}
	\label{hn}
\end{equation}

\noindent where $\rm{Nu_f}$ is the Nusselt number for free convection, the parameters a and b depend on the geometry and the flow regime (\cite{Cengel2005}) and the Rayleigh number is defined as the strength of the thermal buoyancy against the viscous and thermal diffusion.

%The Rayleigh number is defined as the product of the Grashoff and the Prandtl numbers, shown by Eq. (\ref{Ra}):
%
%\begin{equation}
%	\mathrm{Ra = Gr \Pr = {\frac{{g{\beta_{th}}}}{{\nu_{air}^2}}{L^{3}_c}\Delta T^{*}\Pr}}
%	\label{Ra}
%\end{equation}
%
%\noindent where: $\rm{L_{c}}$ is the characteristic length; the volume expansion coefficient $\beta_{\rm{th}}$ is 1/$\rm{T_{air}}$ as the air is considered an ideal gas; $\rm{\Delta T^{*}}$ is the temperature difference between the surface and the air; and $\nu_{\rm{air}}$ is the air kinematic viscosity.

\citet{AbdelKhalik1978} evaluated the natural convective coefficients between the absorber surface and cover plate for vertically oriented two-dimensional CPC using finite-element techniques. The critical Rayleigh number values for different concentration ratios (2 $<$ CR $<$ 10) with three levels of truncated CPCs are determined. Results show that convection is suppressed in high-concentration cavities with a large height/aperture ratio.

%Mathematical formulations were developed to study thermal processes in a compound-parabolic- concentrator (CPC) collector. The system under investigation consists of a CPC cusp fitted with a concentric, evacuated double pipe to serve as a heat absorber. Heat is transmitted to the circulating fluid flowing inside a U-tube via the heat getter slipped inside the inner pipe (\cite{Hsieh1981}).

\citet{Prapas1987} developed a heat transfer model and found out that the Nusselt number is affected by the concentration ratio of the collector, its inclination and the absorber temperature. The average Nusselt number rises as the inclination of the collector increases, and this effect is enhanced for higher concentration ratios. Moreover, generalised correlations for the variation of the Nusselt number have been obtained.

\citet{Eames1993} performed a detailed parametric heat transfer analysis in untruncated CPCs using a model for their optical and thermal behaviour. The effects of inclination and acceptance angles on free convection within the cavity were studied. A convective heat transfer correlation is obtained for the average Nusselt number concerning the Grashof number that considers acceptance angle and angular inclination. They also observed that CPCs of higher acceptance angles have lower Nusselt numbers.

A theoretical and experimental investigation into the modifications in optical and thermal performance resulting from introducing a baffle into the cavity of a CPC has been performed. Results show that introducing a baffle reduces internal convection, reducing heat losses with a slight reduction in optical efficiency (\cite{Eames1995}).

Heat transfer modelling in CPCs has been investigated. This considered the effect of the inclination angle of an east-west aligned collector. The internal and external convective heat transfer correlations employed are angular dependent. The model also considered the contribution of beam and diffuse radiation. The results demonstrate a 10\% variation in convective heat transfer with an angle of inclination for low-concentration CPCs (i.e. CR = 1.5). Furthermore, the thermal efficiency was lower for incoming radiation of more diffuse components. In this case, because of the high diffuse radiation, a CPC of lower concentration would be preferred to maximise the fraction of the diffuse insolation collected. (\cite{Kothdiwala1995}).

The results show that when radiation is neglected, the onset of fluid motion is delayed by the cavity's concentration level. When radiation is considered, it has an important effect on the temperature distribution inside the parabolic cavities and the local and average values of the convective and radiative Nusselt numbers. The emissivity substantially affects the average radiative Nusselt number, especially at high Rayleigh numbers (\cite{Diaz2008}).

\citet{Tchinda2008} developed a mathematical model for computing the thermal performance of a SAHC with truncated CPC having a flat one-sided absorber. The effects of the air mass flow rate, the wind speed and the collector length on the thermal performance of the present collector were investigated. Predictions for the performance of the SAHC also exhibit reasonable agreement, with experimental data with an average error of 7\%.

\subsubsection{Asymmetric CPC}

Symmetric CPCs have two equal half-acceptance angles concerning the optical axis. The asymmetric compound parabolic concentrator (ACPC) introduced by \citet{Rabl1976} is a particular case of its symmetric counterpart. Figure \ref{ACPC} shows a general cross-section of an ACPC, where the acceptance angle is $\rm{2\theta_{acc} = \theta_{\!_{PU}} + \theta_{\!_{PL}}}$. The geometric concentration ratio is also the ratio of the aperture to absorber areas. The axis of the upper (lower) parabola subtends an angle $\rm{\theta_{\!_{PU}} (\theta_{\!_{PL}})}$ perpendicular to the absorber surface. Therefore, broader angular acceptance range and designs with higher concentration ratios can be achieved due to the asymmetry (\cite{Tian2018}). 

\Figure[scale=0.60,placement=!ht,label={ACPC},caption={General design of an asymmetric CPC.}]{figs/ACPC.png}

Asymmetric concentrator systems present the following advantages (\cite{Mills1978}):

\begin{itemize}
	\item Ability to compensate lower solar radiations in the early morning and late afternoon, allowing more uniform output;
	\item Greater operational flexibility for unexpected variations in energy demand and higher yearly average energy input per reflector surface area.
\end{itemize}

Researchers have studied this type of concentrator in detail. \citet{Zacharopoulos2000} analysed the optical performance of a 3D dielectric ACPC with a 78\% truncation at the vertical compared to a symmetric version (Figure \ref{ACPCzac}). The analysis showed that the asymmetric concentrator design is more suitable for use in a building fa\c{c}ade than a symmetric one. Using a dielectric concentrator, an ACPC can collect 40\% of solar radiation with, due to refraction, collection even outside the angular acceptance range. \citet{Tripanagnostopoulos2000} proposed a collector design based on a truncated asymmetric CPC reflector consisting of a parabolic and a circular part. This design features a flat bifacial absorber installed at the upper part of the collector, parallel to the glazing, to form a thermal trap space between the reverse absorber surface and the circular part of the mirror. The experimental results showed that the proposed collector could achieve a maximum efficiency of 71\% and a stagnation temperature of 180 $^{\rm{o}}$C. \citet{Mallick2006} presented a comparative experimental characterisation of a non-imaging line-axis 0~--~50$^{\rm{o}}$ acceptance-half angles asymmetric compound parabolic photovoltaic concentrator (ACPPVC-50) suitable for vertical building fa\c{c}ade integration with its non-concentrating counterpart. \citet{Mallick2007b} performed an optical and heat transfer analysis for a truncated ACPC of concentration ratio 2.01 suitable for photovoltaic applications to use airflow to alleviate temperature at the solar cells. \citet{Sarmah2011} compared the optical performance of three dielectric ACPC designs (all truncated with a concentration ratio of 2.82) of acceptance angles 0 -- 55$^{\rm{o}}$, 0 -- 66$^{\rm{o}}$, and 0 -- 77$^{\rm{o}}$ to optimize the concentrator for building facade photovoltaic applications in northern latitudes ($>$ 55 $^{\rm{o}}$N). Based on the annual solar energy collection by all the designs, it was found that the system of acceptance angles 0 -- 55$^{\rm{o}}$ is more optically efficient and can collect more energy compared to the other two. \citet{Harmim2012} constructed and evaluated the performance of a box-type solar cooker equipped with an ACPC of concentration ratio 2.12. The reflectors were designed so that the absorber could receive solar rays at a solar altitude angle between 30 and 75$^{\rm{o}}$.

\Figure[scale=0.60,placement=!ht,label={ACPCzac},caption={Dielectric ACPC with 78\% truncated at the vertical for building integration photovoltaic.}]{figs/ACPCzac.PNG}

%\Figure[scale=0.70,placement=!ht,label={cooker},caption={Sketch of the box-type solar cooker employing an ACPC.}]{figs/cooker.PNG}

\subsubsection{ACPC with Inverted Absorber}

Collectors employing inverted absorbers, in which solar radiation is reflected from below onto the downward-facing absorbing surface, have been proposed by \citet{Rabl1976} and shown in Figure \ref{col_rev}. They are also called inverted absorber asymmetric compound parabolic concentrators (IACPC). Although optically less efficient due to the multiple reflections of incident solar energy (\cite{Eames1996}; \cite{Kothdiwala1996}; \cite{Shams2013}), this type of concentrator can achieve higher absorber temperatures by suppressing convective and radiative heat losses (\cite{Kothdiwala1997}; \cite{Kothdiwala1999}). This is due to the formation of thermally stratified air layers below the absorber and also because this surface does not view the aperture directly (\cite{Kienzlen1988}; \cite{Eames2001}).

\Figure[scale=0.50,placement=!ht,label={col_rev},caption={Basic CPC geometry with inverted absorber.}]{figs/col_rev.eps}

%Researchers have reported studies aiming to evaluate the performance of this type of collector. 
\citet{Kothdiwala1996} developed a ray trace model to simulate and optimise the IACPC optical performance. This model considered the effect of beam and diffuse radiation separately. They verified that the beam optical efficiency decreases with the cavity height and concentration ratio increase. It was also found that the diffuse optical efficiency is improved when the acceptance angle is increased.

%\Figure[scale=0.70,placement=!ht,label={koth96},caption={Geometry of the CPC with inverted absorber analysed.}]{figs/koth96}

\citet{Eames1996} predicted the thermo-physical performance of the IACPC system. In their study, the energy flux at the absorber was determined by a ray trace technique and a finite element model was developed to predict the system's thermo-fluid behaviour. They concluded that a net gain in efficiency is achieved by including a cavity above the circular reflector due to the convection suppression.
\citet{Kothdiwala1997} conducted indoor experiments under a solar simulator to analyse the performance of the IACPC, which was copper sheeting onto a tubing along the concentrator's long axis. The tests were carried out using water as the flowing fluid at various cavity heights. They found that the overall performance is more efficient for higher gap height configurations.
\citet{Kothdiwala1999} compared a tubular absorber CPC with a glass envelope to an IACPC at different cavity heights for water heating purposes. They concluded that using an absorber configuration on the IACPC maximises convection suppression and minimises optical losses. Furthermore, this system outperforms the others in terms of optimum configuration compared to this study.
\citet{Eames2001} simulated the performance of an IACPC by using a combined ray trace and finite element computational fluid dynamics model previously developed by \citet{Eames1993a}. This model was validated by direct comparison with experimental results.

\citet{Tiwari1998} modelled and evaluated the performance of an inverted absorber solar still for distillation purposes. They found that this inverted configuration provided double the hourly yield compared to a conventional still. The experimental comparison between inverted absorber solar still and conventional single slope solar at various water depths has been conducted by \citet{Dev2011}. They found that the water temperature in an inverted absorber solar basin is still higher than the conventional one.

In order to suppress convection losses, \citet{Smyth2005} investigated the use of transparent baffles at different locations within the collector cavity; the system consisted of an integrated collector storage solar water heater (ICSSWH) mounted in the cavity of an IACPC. \citet{Shams2016} designed and fabricated a concentrating transpired air heating system comprised of an IACPC with a perforated absorber. This collector had the transpired absorber surface made of woven carbon fibre placed at a fixed cavity height, a glazed aperture, a concentration ratio of 2.0, and was experimentally tested at different air flow rates.

\subsection{Optical systems and Ray tracing technique}

A significant part of the design and analysis of concentrating collectors involves ray tracing techniques, which are algorithms to simulate light rays passing through an optical system. Ray tracing analysis is an important method adopted in optical systems to obtain the optical performance for complex geometries regarding direct and diffuse solar radiation (\cite{Ali2013}). Most of its energy will be reflected when a ray hits a reflecting surface. To model this behaviour in a suitable ray tracing procedure, the law of reflection is expressed in vector form (\cite{Winston2005}). Figure \ref{ref_point} shows the unit vectors $\rm{r_{inc}}$ and $\rm{r_{ref}}$ along the incident and reflected rays and a unit vector $\rm{r_n}$ at the normal point of incidence into the reflecting surface. The law of reflection is expressed by Eq. (\ref{ref_law}):

\begin{equation}
	\mathrm{{r_{ref}} = {r_{inc}} - 2({r_n} \cdot {r_{inc}}){r_n}}
	\label{ref_law}
	\end{equation}

\Figure[scale=0.80,placement=!ht,label={ref_point},caption={Law of reflection applied on a reflecting surface.}]{figs/ref_point.eps}

The ray tracing analysis with optical study can provide:

\begin{itemize}
	\itemsep-5pt
		\item The average number of reflections before the incoming rays reach the absorber plate (\cite{Shams2013}; \cite{Benrejeb2016});
		\item Optical efficiency as a function of the incidence angle (\cite{Kothdiwala1996}; \cite{Souliotis2011});
		\item Visualisation of rays' path and reflection points (\cite{Mallick2007}; \cite{Ratismith2014}; \cite{Ustaoglu2016});
		\item The intensity of energy distributed at the absorber surface (\cite{Smyth1999}; \cite{Sellami2013}; \cite{Ali2014}; \cite{Bellos2016});
		\item System's optical characterisation for thermal modelling and simulation (\cite{Mallick2007}; \cite{Shams2013}; \cite{Bellos2016});
		\item Comparison between two or more systems (\cite{Zacharopoulos2000}; \cite{Sarmah2011}; \cite{Wu2009}).
	\end{itemize}

Several concentrating systems have been proposed and optically analysed for different purposes and reported in the literature in detail. \citet{Souliotis2011} used a two-dimensional ray tracing method to analyse the optical properties of an asymmetric CPC collector. The process involved tracing the paths of many rays through the system and calculating the acceptance angle. The results showed that the collector achieves optical efficiencies above 75\% within its acceptance angle, decreasing efficiency rapidly outside this range.

\citet{Sarmah2011} presented the design and optical performance evaluation of stationary dielectric asymmetric compound parabolic concentrators using ray tracing methods. The designed concentrators have a geometric concentration ratio of 2.82 and a maximum optical efficiency of 83\%. The ray tracing simulations show that all rays within the acceptance half-angle range can be collected without escaping from the concentrator's aperture.

\citet{Zheng2011} presented a new multiple chamber trough solar collector, and optical analysis software was used to simulate the ray tracing of the solar light concentrating system. The study investigated the flat and cylindrical receivers and the relationship between the receiving beam and the incident ray. The simulation results showed the image's distribution, width, eccentric magnitude, and efficiency of the concentrated light varying with the incident angle. The concentration ability of the system with a flat receiver and a cylindrical receiver was quantitatively analysed and compared.

Using the OpticsWorks software, \citet{Sellami2012} performed an optical analysis. They developed a novel geometry of a 3D static concentrator as a square elliptical hyperboloid (SEH) to be integrated into glazing windows or fa\c{c}ades for photovoltaic applications. The SEH of concentration ratio 4.0 was optically optimised considering different incident angles of the incoming light rays.

\citet{Ali2013} evaluated the optical performance of a static 3D elliptical hyperboloid concentrator using a ray tracing software called Optis. Effective concentration ratio, optical efficiency and geometric parameters were analysed. Furthermore, the geometry was optimised to improve the overall performance.

\citet{Binotti2013} proposed an analytical approach to evaluate the impact of 3D effects on the optical performance of parabolic trough collectors. The approach extends the First-principle OPTical Intercept Calculation (FirstOPTIC) method and was validated against numerical solutions and ray-tracing simulation results. The new approach was applied to case studies to examine the impact of 3D effects on the intercept factor, and a correction was proposed for the approach generally accepted for specularity mirror errors for non-zero incidence angles.

\citet{Sellami2013} developed a 3D ray trace code in Matlab to determine the beam optical efficiency and the energy distribution of a 3D crossed CPC (CCPC) for different incident angles. The authors found that this type of CPC is an ideal concentrator for a half-acceptance angle of 30$^{\rm{o}}$ and a concentration ratio of 3.6.

\citet{Ali2014} presented the design and experimental analysis of a 3D solar elliptical hyperboloid concentrator (EHC) for process heat applications. Ray tracing analysis was used to obtain the solar flux distribution on the receiver aperture plane, and the optical efficiency was obtained theoretically using a ray tracing program. The design was optimized before finalizing and experimentally testing the EHC.

\citet{Abu-Bakar2014} proposed a new type of concentrator, the rotationally ACPC, for use in building integrated systems for PV applications, where the geometrical and optical concentration gain were evaluated. From the simulations, it has been found that the concentration could produce an optical concentration gain as high as 6.18 when compared with the non-concentrating cell, depending on the half-acceptance angle.

\citet{Ratismith2014} proposed non-tracking configurations of solar collector modules designed to operate efficiently during the day for varying incident angles of direct and diffuse radiation. The design criteria for achieving a high intercept factor without tracking throughout the day are emphasised by conducting ray tracing analysis on different trough shapes and absorber plate orientations. Furthermore, the superiority of the flat base-collector over the double-parabolic design was demonstrated.

\citet{Abdullahi2015} investigated the optical efficiency of two tubular receivers in a compound parabolic concentrator or a single elliptical receiver. Ray tracing is used to predict the optical efficiency, and the results show that the horizontal configuration outperforms the single and vertical configurations by up to 15\%. Moreover, the horizontally aligned elliptical single-tube configuration increases the average daily optical efficiency by 17\% compared to the single-tube configuration.

\citet{Benrejeb2015} used mathematical equations describing the geometric design of an integrated collector storage system. Therefore, an optical study was given with details on achieving the ray tracing technique results and the energy flux distribution on the absorber surface. Furthermore, the optical results were used as inputs in the heat transfer model to simulate the temperature of the water inside the absorber.

\citet{Benrejeb2016} worked on a numerical model based on a ray tracing technique to study the effect of truncation on the optical and thermal performances of an integrated collector storage system of solar water heaters with asymmetric CPC reflectors. The model can predict both full and truncated CPC systems and the simulation involves analysing several parameters such as geometric concentration and half acceptance angle. The effect of truncation on ray trace diagrams and its impact on optical and thermal performances was also studied.

\citet{Bellos2016} performed an optical analysis and optimised the geometry of a CPC with an evacuated tube, where this design is considered to be optimum because all the reflected rays reach the receiver. They also calculated the optical losses at different solar angles. The authors also indicate the need to track the collector to minimise the incident angle. \citet{Qin2013} designed and optimised the geometry of an aspheric reflecting solar concentrator to focus sunlight on a narrow line segment. They used a particular aspheric equation in three dimensions and the law of reflection to trace the incident rays.

\citet{Ustaoglu2016} developed an optical analysis on a cylindrical CPC to reduce hot spots on a PV cell caused by non-uniform solar irradiation. Different truncation levels were tested to determine the optimum optical and thermal efficiency levels. The study analysed average efficiency, incident angle, and annual performance for different absorber surfaces. Heat flux and temperature distribution on the absorber were also evaluated to determine the uniformity of solar illumination.




\section{Building integration of solar systems}



%In addition to being technically and structurally efficient, solar thermal collectors must satisfy criterion summarised in the IEA Task 41 Solar Energy and Architecture for aesthetic quality of buildings integrated solar thermal collectors [4]:

%\begin{itemize}[topsep=5pt,partopsep=0pt] \itemsep0pt
%	\item Integrating naturally; 
%	\item Architecturally pleasing design;
%	\item Consistency to the context of the building;
	%\item Size that suits the harmony and combination;
%	\item Good composition of colours and materials;
%	\item Well composed and innovative design.
	
%\end{itemize}

A building-integrated solar thermal system (BISTS) is a solar thermal collector integrated into a building to meet local energy requirements. This integration must consider functionality (useful thermal energy, thermal insulation, shading, construction stability) and/or appearance aspects (aesthetics, dimensions, shape, colour of the building). In addition to that, they need to be technically and structurally efficient (\cite{Wall2012}; \cite{Buker2015}). A few more factors will need to be taken into consideration, such as (\cite{COSTOffice2015}): i) amount of useful thermal energy collected and fluid temperature delivered; ii) resistance to weather conditions; iii) light and solar energy characteristics in case of transparent layer; iv) thermal resistance and thermal transmittance characteristics of the construction (overall heat transfer coefficient); v) fire protection, and; vi) noise attenuation.
 
% coupled onto the exterior of a building -- usually mounted on the fa\c{c}ade or on the roof -- to meet local energy requirements.



% These systems' designs must consider architectural aspects, such as colour, materials, texture and shape, enabling a more homogeneous building aesthetic than conventional solar thermal collectors. Plus, they need to integrate harmony with the building, be consistent with the building context, and be of well composed and innovative design. In addition to that, they need to be technically and structurally efficient (\cite{Wall2012}; \cite{Buker2015}).

%Building integrated solar thermal system: We consider STS as building integrated, when some components (mainly the solar thermal collector) is an integral part of the building functionality, not just an added element. This integration may be functional (i.e. thermal insulation, shading, construction stability etc. will be compromised) and/or relates to the appearance (aesthetics, dimensions, shape, colour etc. of the building). Thus building integration considers architectural integration in form and function.

These systems have been classified across operating characteristics, features, and mounting configurations. The main classification criteria of all solar thermal systems are based on the method of transferring collected solar energy to the application (active or passive), the thermal transfer fluid (air, water, water-glycol, oil, etc.) and the final application for the energy collected (hot water and/or space heating, cooling, process heat or mixed applications). In the passive or active classification, in the first case, the thermal transfer fluid flows by natural convection or circulation or no transport at all. In the second case, pumps or fans circulate the fluid to a point of demand or storage (forced convection or circulation). However, several systems are hybrids, operating by natural and forced transport methods. Many fa\c{c}ade solar air heaters use thermal buoyancy to induce airflow through the vertical cavities that can be further augmented with in-line fans (and heating) if necessary.

The BISTS delivers thermal energy to the building, but other forms of energy may also contribute to the building's energy balance. For instance, daylight comes through a transparent window or fac¸ade collector, or PV/T systems will also deliver electrical power, which may be used directly by any auxiliary electrical services. Heated air or water can be stored or delivered directly to the point of use. Although the range of applications for thermal energy is extensive, all of the evaluated studies demonstrate that the energy is used to provide one or a combination of the following cases:

\begin{description}
	\item[Space heating:] Thermal energy produced by a BISTS may reduce the space heating load of a building by adding solar gains directly (by a passive window) or indirectly (by transferring heat from the collector via storage to a heating element) into the building;
	
	\item[Air heating and ventilation:] Thermal heat may also be used to pre-heat the fresh air needed in the building. Air is heated directly or indirectly to provide heating and/or ventilation space. In some cases, an auxiliary heating system is used to enhance the heat input because of comfort reasons;
	
	\item[Water heating:] Hot water demand in the building is the most popular application. In the majority of water heating BISTS, a customised heat exchanger or integrated proprietary solar water is used to transfer collected heat to a (forced) heat transfer fluid circuit and onto an intermediate thermal store and/or directly to a domestic hot water application; 
	
	\item[Cooling and ventilation:] In cooling-dominated climates, buildings may have an excess of thermal energy, and therefore, BISTS can also be a technology to extract heat from a building. Several methods are described for providing a cooling (and/or ventilation) effect to a building: shading vital building elements, desiccant linings, induced ventilation through a stack effect and reverse operation of solar collecting elements for night-time radiation cooling.
\end{description}

%Air-based BISTs are characterised by lower costs and efficiencies than water heating systems. They are generally used for building ventilation and heating. Hydraulic BISTs are generally used for heating and hot water generation. BISTs that adopt phase change materials are generally used when a longer period of the system output is required. These systems can be categorised in function of the integration with the building envelope, as shown in Figure 9. For many decades integrated solar thermal systems have been studied with research efforts focused on different technological solutions to BIST integration.
%
%Air-based BISTs are solar thermal air collectors integrated on roofs and facades, as shown in Figure 10. These collectors are characterised by low costs but also by a low efficiency due to the thermal characteristics of the air as a heat transfer fluid. Air has low thermal properties, thus influencing convective heat transfer phenomena. To compensate these limits, large collector areas and ducts are needed. However, these can introduce problems related to costs and roof/facade size [23]. Solar thermal facades that use air can be made by integrating an air gap between the rear surface of the glass covers and the building envelope. In space heating systems, constant air flows are generally provided, thus the outlet air temperature varies during the day in function of the solar radiation variations.
%
%Using water instead of air is effective because of its high thermal capacity and thermal conductivity. Water allows easy storage as it is suitable for direct domestic hot water generation, but it is corrosive. %The water-based BIST can be classified into two groups: single-channel and multiple-channel BISTs. The first group has evolved from the passive solar heating mechanism, which often provides solutions for the use of hot water. The second group is characterised by a design with an additional air space between the photovoltaic module and the building envelope.
%
%Roof integrated mini-parabolic solar collector have been studied for building-integrated applications. These systems adopt linear Fresnel reflectors to focus the beam solar radiation on a stationary receiver with the use of mirrors and an active tracking system (ref). The use of concentrating optical system for building integrated solar applications was proposed by Chemisana et al. [25]. The analysis of a stationary wide-angle Fresnel lens with a moving CPC was proposed by the authors. A building integrated mini parabolic thermal collector investigated by Petrakis et al. [26], had a East-West orientation with a slope to receive the Figure's rays perpendicularly during the day. A solar micro-concentrator collector is shown in Fig. 11.
%
%Another solution is represented by the ceramic solar collectors, mainly characterised by economic convenience, without absorption attenuation and good integration in buildings. As asserted by Buker and Riffat [23], the raw material for this system is traditional ceramic (porcelain clay, quartz and feldspar). The building integration of these solar collectors was also proposed by Yang et al. [27]. The collectors act both as heat source of the water system, and as balcony railings, with a thermal efficiency equal to 41.7\%.
%
%Overhang louvre shading modules, placed horizontally, can incorporate solar collectors. This approach allows an increased number of collectors to be installed as space is freed on the building roof for the installation of additional panels. The various designs of solar louvre thermal collector have been discussed by Abu-Zour et al. [28]. They have proposed a design that used heat pipe technology. Marrero et al. [29] analysed a solar thermal system that exploited building louvre shading devices. They proposed the modification of existing designs. The proposed system was tested under several climatic conditions, showing great potential from both an economic and a renewable energy supply point of view.
%
\subsection{Building integrated solar thermal collectors for air heating}

These collectors are known for being cost-effective; however, they also have lower efficiency. This reduced efficiency is mainly due to the thermal properties of air, which is used as the heat transfer medium. The low thermal conductivity of air significantly affects the convective heat transfer within these systems. To overcome these efficiency challenges, it is necessary to implement larger collector areas and duct systems. However, deploying such measures may engender cost escalation and impose spatial constraints on building surfaces. Notably, solar thermal facades employing air as the heat transfer medium can be engineered by incorporating an air gap between the rear surface of glass covers and the building envelope. In space heating applications, consistent air flows are typically maintained, resulting in fluctuations in outlet air temperature throughout the day in response to variations in solar radiation intensity.

%fa\c{c}ade

%Using water instead of air is effective because of its high thermal capacity and thermal conductivity. Water allows easy storage as it is suitable for direct domestic hot water generation, but it is corrosive. The water-based BIST can be classified into two groups: single-channel and multiple-channel BISTs. The first group has evolved from the passive solar heating mechanism, which often provides solutions for the use of hot water. The second group is characterised by a design with an additional air space between the photovoltaic module and the building envelope.

%Roof integrated mini-parabolic solar collector have been studied for building-integrated applications. These systems adopt linear Fresnel reflectors to focus the beam solar radiation on a stationary receiver with the use of mirrors and an active tracking system (ref). The use of concentrating optical system for building integrated solar applications was proposed by Chemisana et al. [25]. The analysis of a stationary wide-angle Fresnel lens with a moving CPC was proposed by the authors. A building integrated mini parabolic thermal collector investigated by Petrakis et al. [26], had a East-West orientation with a slope to receive the Figure's rays perpendicularly during the day. A solar micro-concentrator collector is shown in Fig. 11.

Mini-parabolic solar collectors integrated into roofs have been explored for building integration purposes. These systems utilise linear Fresnel reflectors to concentrate solar radiation onto a fixed receiver by manipulating mirrors and an active tracking mechanism. The study proposed a stationary wide-angle Fresnel lens coupled with a moving CPC. Additionally, a building-integrated mini-parabolic thermal collector was studied in an East-West direction with a slope optimised to intercept sunlight perpendicularly throughout the day.

%Another solution is represented by the ceramic solar collectors, mainly characterised by economic convenience, without absorption attenuation and good integration in buildings. As asserted by Buker and Riffat [23], the raw material for this system is traditional ceramic (porcelain clay, quartz and feldspar). The building integration of these solar collectors was also proposed by Yang et al. [27]. The collectors act both as heat source of the water system, and as balcony railings, with a thermal efficiency equal to 41.7\%.

Ceramic solar collectors offer an alternative solution, notable for their economic viability, minimal absorption attenuation, and seamless integration into building structures. These collectors use conventional ceramic materials such as porcelain clay, quartz, and feldspar. It was also suggested that these solar collectors be integrated into buildings. In this proposed application, the collectors serve dual purposes: providing heat for water systems and serving as balcony railings, achieving a thermal efficiency of 41.7\%. 

%Overhang louvre shading modules, placed horizontally, can incorporate solar collectors. This approach allows an increased number of collectors to be installed as space is freed on the building roof for the installation of additional panels. The various designs of solar louvre thermal collector have been discussed by Abu-Zour et al. [28]. They have proposed a design that used heat pipe technology. Marrero et al. [29] analysed a solar thermal system that exploited building louvre shading devices. They proposed the modification of existing designs. The proposed system was tested under several climatic conditions, showing great potential from both an economic and a renewable energy supply point of view.

Horizontal overhang louvre shading modules offer a space-efficient platform for integrating solar collectors, enabling the installation of more collectors by freeing up roof space for additional panels. It explored various designs of solar louvre thermal collectors, introducing a design incorporating heat pipe technology. Similarly, a solar thermal system utilising building louvre shading devices and suggested modifications to existing designs were investigated. Testing the proposed system under diverse climatic conditions revealed significant potential both economically and in terms of renewable energy supply.

\section{Chapter summary}

This chapter depicted a comprehensive literature review on solar air heating collectors (SAHCs) and related solar thermal technologies, highlighting their design, influencing factors, and diverse applications. SAHCs are widely used for industrial drying, space heating, and agricultural purposes, with two primary types: unglazed collectors, featuring perforated metallic plates without glazing to enhance air contact, and glazed collectors, which include glazing to minimise heat loss and improve efficiency. Key performance factors include the absorber surface, where material conductivity, coatings with high absorptivity and low emissivity, and advanced geometries like finned or V-corrugated designs play a significant role; the glazing cover, which reduces heat losses while balancing solar transmissivity and infrared reflectivity; and airflow rate, which affects thermal efficiency and outlet temperature. System modifications, such as single or double air passes, channel dimensions, and the configuration of multiple collectors in series or parallel, further influence performance. 

Concentrating collectors, including parabolic troughs, compound parabolic concentrators (CPCs), and heliostat fields, enhance thermal output by focusing solar radiation onto smaller absorber areas, with optical performance assessed through parameters like incident angles, glazing optics, and reflector geometry. 

Building-integrated solar thermal systems (BISTS) seamlessly incorporate collectors into structures like rooftops and facades, addressing functionality (thermal energy delivery, space heating, or cooling) and aesthetics. Despite challenges such as the low thermal conductivity of air, innovations like ceramic collectors, optimised shading modules, and advanced geometries improve efficiency. These systems also offer versatility in applications, including water heating, ventilation, and cooling, making them integral to sustainable building designs.






%
%\subsubsection{Air heating}
%
%The solar air collectors are widely applied in many commercial applications such as hot air supply to shopping malls, agricultural barns and industrial drying etc. They are usually low cost, with no freezing and high pressure problems. However, one of the main disadvantages of solar thermal air collectors is their relatively low efficiency due to the low density, volumetric heat capacity and thermal conductivity of air causing to low coefficients of convective heat transfer from solar energy to the air. In order to compensate this drawback, air needs to be encapsulated in larger collector areas which puts cost and roof size problem in front [55]. A schematic of the working principle of a roof integrated solar thermal air collector is illustrated in Fig. 18. 
%
%The solar energy absorbing uppermost layer may have a solar selective coating and internal duct configuration to increase heat transfer ratio from the heated absorber layer to the internal air stream. A flat transpired solar thermal air collector is formed an unglazed, perforated solar absorbing layer. The air stream heated at the layer is pulled through an array of small perforations by a fan. A large proportion of the heated air, that would otherwise compose convective heat loss from the heated surface, is thus loaded into the system. At relatively high flow rates and low supply temperatures, the solar thermal air collector system can display high solar conversion efficiency [57]. A schematic diagram of solar thermal air collector application is shown in Fig. 19.
%
%\subsubsection{Water heating}
%
%Solar water heating collectors prove to be an effective concept for conversion of solar energy into thermal energy. The efficiency of solar thermal conversion reaches up to 70\% when compared to direct conversion of solar electrical systems which have an efficiency of around 17\%. Thus solar water heating collectors play a crucial role in domestic use as well as industrial sector due to its ease of operation. Flat plate collector is the pivotal component of any solar water heating system. Thermal performance characteristics of a flat plate solar water heater mainly depend on the transmittance, absorption and conduction of solar energy and good conductivity of the working fluid [65]. The cross sectional view of a flat plate solar water collector and the schematic view of a typical thermosyphon solar water heating system is shown in Fig. 21.
%
%\subsubsection{Photovoltaic/thermal}
%
%A photovoltaic/thermal (PV/T) collector is a combination of photovoltaic (PV) and solar thermal components that produce both electricity and heat simultaneously. This dual function of the PVT enables a more effective use of solar energy that results in a higher overall solar conversion. Much of the captured solar energy in a sole PV module elevates the temperature of its cells which causes degradation in module efficiency. This waste heat needs to be removed to ensure a high electrical output. The PV/T technology recovers part of this extracted heat to utilise for low-and-medium-temperature applications [6]. 
%
%The merits of PV/T concept comparing to alternative technologies contain eco-friendly, proven long life (20–-30 years), noise free and low maintenance. However, several factors restrict the efficiency of the photovoltaic module such as particularly temperature increase and utilising only a part of solar spectrum (photon energy threshold is less than 1.11 μm for c-Si) for power generation. The band gap of silicon (1.12 eV) limits the total energy collected in solar spectrum even less than 1.11 μm. So, photons of longer wavelength dissipate their energy as waste heat rather than generating electron–hole pairs. As PV modules are able to convert only 4–17\% of the incoming solar radiation into energy depending on the solar cell type and working conditions, cooling PV modules simultaneously by a fluid stream like air or water boost energy yield significantly. Conceptually, re-use of heat energy extracted by the coolant is ideal. Thus, PV/T collectors offer a higher overall efficiency [7].



