\chapter{Introduction}
\label{Cap:Int}

To achieve the carbon emission target, high fraction of locally available renewable energy sources will become necessary to reduce energy demand. Solar energy is one of the most important renewable sources for use in building heating, cooling, hot water supply and power production. Solar thermal systems for water and air heating are well established technologies. Their use promotes environmental sustainability, as it eliminates the need to provide heat by combusting of fossil fuels therefore contributing to reducing greenhouse gases emissions. Building integrated solar thermal systems (BISTS) can be a potential solution towards the enhanced energy efficiency and reduced operational cost in contemporary built environment. According to the vision plan issued by European Solar Thermal Technology Platform (ESTTP), by 2030 up to 50\% of the low and medium temperature heat will be delivered through solar thermal systems (\cite{ESTTP2009}).

Solar air heating collectors transfer solar thermal energy from an absorbing surface to an airflow. The heated air can be used for applications, such as space heating, timber seasoning, curing of industrial products and food drying (\cite{Shams2016}), as can be seen in Table \ref{air-application} with the corresponding air temperature range. A solar air heater can integrate to a concentrator to produce air at medium or high temperatures (\cite{Duffie2013}). A concentrating collector combining inverted absorber and asymmetric compound parabolic concentrator (IACPC), in which solar radiation is reflected to be incident from below onto the absorbing surface, has been proposed by \citet{Rabl1976}. Although there are optical losses due to the multiple reflections of incident solar energy (\cite{Kothdiwala1996}), this type of concentrator is able to achieve higher absorber temperatures by suppressing convective and radiative heat losses (\cite{Kothdiwala1999}). A competing technology for this application is on evacuated tube collectors (ETC). This also has optical losses due to reflection from the tubular aperture. An ETC's low heat losses arise from a vacuum between the glass tube envelope and an absorber. As well as limiting long-term durability, this is challenging to fabricate and to envelope the large cross-section duct required for airflow. 

\begin{table}[!ht]
	\caption{Application for air heating and operating air temperature range. From \citet{Norton2012} and \citet{Pranesh2019}}
	\centering
	\begin{tabular}{p{4cm}p{4cm}}
		\hline \\[-10pt]
		Application & Temperature range ($^{\rm{o}}$C) \\
		\hline \\[-10pt]
		Timber drying & 60 -- 100 \\ [3pt]
		Food drying & 30 -- 90 \\ [3pt]
		Brick curing & 60 -- 140 \\ [3pt]
		Space heating & 30 -- 100 \\ [3pt]
		\hline
	\end{tabular}
\label{air-application}
\end{table}

An IACPC was designed to be used as a solar air heater collector by \citet{Shams2013}. This collector had a perforated absorber surface made of carbon fibre placed at a fixed cavity height, a glazed aperture, a concentration ratio of 2.0, and was optically characterised and experimentally tested at different airflow rates. In this design, the air inlet was located at one end below the absorber and the outlet was placed at the top near the other end. The lowest experimental airflow rate (0.03 kg/(m$^2$.s)) resulted in the highest temperature rise of 38 $^{\rm{o}}$C at a radiation level of 1000 W/m$^2$. The highest experimental operating flow rate (0.09 kg/(m$^2$.s)) resulted in a temperature rise of 19.6 $^{\rm{o}}$C at a radiation level of 1000 W/m$^2$.

%This current research examined IACPC collectors stacked on a wall. The collector had a vertical aperture so that shading could be avoided when two or more collectors are stacked on a vertical wall, enabling the full façade area available to be harnessed (as shown in Figure \ref{array1}. The concentration ratio was increased to receive more solar radiation. One collector was fabricated then tested experimentally using five different air flow rates (0.04 -- 0.12 kg/(m$^2$.s)). Experimental results show that, on clear days and in periods of solar insolation greater than 800 W/m$^2$, the airflow temperature delivered varied between 12 $^{\rm{o}}$C and 27 $^{\rm{o}}$C above ambient temperature.

This current research designed IACPC collectors to be stacked on a wall. The collector had a vertical aperture so that shading could be avoided when two or more collectors are stacked on a vertical wall, enabling the full façade area available to be harnessed (as shown in Figure \ref{array1}. The concentration ratio was increased to receive more solar radiation. However, solar air heating systems comprising more than one interconnected IACPC collector have not been thermal characterised to date. The number of collectors needed to meet air temperature requirement for specified conditions is not known nor is the cost involved. The behaviour of the airflow and the heat transfer mechanism inside the collector have never been studied in sufficient detail to enable detailed parametric analysis. Such understanding will allow designers to propose improvements to the system. Based on the scale and size of the collector developed, the system has the potential application for pre-heating fresh air for buildings.

\Figure[scale=0.40,placement=!ht,label={array1},caption={Array of 3 collectors stacked on a 1-m$^2$ wall of facade.}]{figs/array1.png}

%One of the ways to address this is to simulate the collector in CFD then test different modifications to assess the system’s thermal performance. The contribution to the knowledge is to study this particular type of collector, propose improvements and learn how to operate the system to provide required airflow temperatures. Based on the scale and size of the collector developed, the system has the potential application for pre-heating fresh air for buildings and solar drying.

\section{Aim and objectives of this research}
This research aims to design and characterise an asymmetric compound parabolic air heating solar collector with an inverted transpired heating absorber. The intended application constraint is a building integrated air heating system comprising several collectors capable of collecting solar radiation for eight hours per day in summer to heat an airflow. An array of 3 collectors is shown in Figure \ref{array1}. The specific objectives are to:

\begin{itemize}
	\item develop a mathematical model to build the collector's shape;
	\item use this model to assess the optical performance with a ray tracing technique in 3D and thus decide the design of the collector;
	\item fabricate a prototype and undertake field tests for the collector characterisation to achieve a well understanding of the operational variables that affect its performance;
	\item simulate the collector's thermal behaviour through a transient heat transfer modelling and validate it using the experimental results obtained from the prototype unit;
	\item simulate the thermal performance of more than one collector connected in series and in parallel from the validated model;
	\item simulate a barley drying as an application to the solar collector system based on the energy demand.

\end{itemize}

\section{Significant new knowledge to the research area}

The significant new contribution to knowledge in the research area of solar thermal energy for air heating are:

\begin{itemize}
	\item development of an accurate model capable of bulding the shape of CPCs, ACPCs and coupled with inverted absorbers. It considers design parameters, such as parabolic shape, concentration ratio, truncation, length, inclination and absorber position;
	\item coupling of this model to a 3D ray tracing technique that simulates direct solar radiation using solar angles (azimuth and altitude) from the Sun's position;
	\item additional results obtained from an optical modelling in 3D when compared to the modelling in 2D, such as end losses and energy distribution along the absorber area;
	\item thermal modelling and simulation of more than one collector connected in series and in parallel to deliver higher airflow temperatures or more useful energy rate;
	\item use of the thermal modelling to simulate barley drying application to find how much energy can be replaced by the solar air heating system.
\end{itemize}

\section{Thesis overview}

%\Figure[scale=0.47,placement=!ht,label={schematics_intro},caption={Research Methodology.}]{figs/schematics_intro.eps}

The research methodology is shown in Figure \ref{scheme-intro}. The Thesis overview with the chapters is depicted as follows:

\begin{figure}[!ht]
	\centering
	\begin{tikzpicture}[scale=1, node distance = 5 cm, very thick]
		\node[block, text width=6cm] (intro) {Introduction and literature review};
		\node[block, below of=intro,text width=3cm,yshift=3cm] (mat) {Materials selection};
		\node[block, below of=mat,text width=3cm,yshift=2cm] (fab) {Fabrication of a prototype};
		\node[block, below of=fab,text width=3cm,yshift=2cm] (exp) {Thermal performance of the prototype};
		\node[block, left of=mat,text width=3cm] (opt) {Optical analysis};
		\node[block, left of=fab,text width=3cm] (opt2) {Optical characterisation and design};
		\node[block, right of=mat,text width=3cm] (thermal) {Thermal modelling};
		\node[block, right of=fab,text width=3cm] (thermal2) {Thermal simulation};
		\node[block, right of=exp,text width=3cm] (vali) {Modelling validation};
		\node[block, below of=exp,text width=5cm,yshift=2cm] (two) {Simulation of more than one prototype connected};
		\node[block, below of=two,text width=4cm,yshift=2.5cm] (dry) {Drying simulation};
		\draw[arrow] (intro) -- (mat);
		\draw[arrow] (mat) -- (fab);
		\draw[arrow] (fab) -- (exp);
		\draw[arrow] (exp) -- (two);
		\draw[arrow] (intro) -| (opt);
		\draw[arrow] (opt) -- (opt2);
		\draw[arrow] (opt2) -- (fab);
		\draw[arrow] (intro) -| (thermal);
		\draw[arrow] (thermal) -- (thermal2);
		\draw[arrow] (thermal2) -- (vali);
		\draw[arrow] (exp) -- (vali);
		\draw[arrow] (vali) |- (two);
		\draw[arrow] (two) -- (dry);
	\end{tikzpicture}
%\label{scheme-intro}
\caption{Research methodology.}
\label{scheme-intro}
\end{figure}

\begin{itemize}
	\item Chapter \ref{Cap:Int} introduces the research topic, research motivation and establishes the problem statement. The main aims of the research, specific objectives and methodology used are presented here;
	\item Chapter \ref{Cap:Lit} -- Literature review: it presents a literature survey that identifies the state-of-the-art of solar thermal collectors for medium temperatures applications with a focus on the main factors that affect the thermal performance. This chapter also presents the optical aspects of a CPC and how an optical analysis can be performed. Optical concentrator designs were examined to identify possible ways to concentrate solar radiation on to an inverted absorber. Lastly, it shows an overview building integrated solar thermal collectors;
	\item Chapter \ref{Cap:Opt} -- Optical modelling and design analysis: presents the design of the solar air heating concentrator as well as the development of a model to calculate optical efficiency. It also depicts details of the concentrator's geometric specification. A 3D ray tracing technique implemented in Matlab software were developed to calculate the energy distribution at the absorber surface and the effect of the end plates on the optical performance;
	\item Chapter \ref{Cap:Exp} -- Experimental performance analysis: it describes physical properties of the materials, fabrication and experimental characterisation of the built collector prototype at different airflow rates. The performance of the system depends on wind speed, solar radiation, inlet air and ambient temperatures;
	\item Chapter \ref{Cap:Thermal} -- Heat transfer modelling and simulation: depicts the heat transfer model developed to characterise and simulate the thermal performance of the proposed solar air heater in operation. This model was used to calculate the outlet airflow temperature and thermal efficiency and experimental data was used to validate the model. Then the model was used to simulate the thermal performance of collectors connected in series and in parallel. Lastly, the validated model was used as an energy input for simulating barley drying to meet the energy demand for an application.
	\item Chapter 6 -- Conclusions: it summarises the main conclusions, presents the contributions to knowledge in the field and then proposes recommendations for future work.
\end{itemize}

%The energy demand is growing quickly, with an excessive use of fossil fuels in civil and industrial sectors. Due to the increasing demand, by 2025 the oil consumption could reach more than 120 million barrels/day [1]. This high employment of fossil fuels cause air pollution and, consequently, global warming [2]. Technologies able to utilise renewable energy are considered sustainable and environment friendly, as they have a lower environmental footprint than the traditional ones. Clean energy for domestic and industrial uses can be obtained from the sun. The total energy received from the sun is a function of many factors, such as geographical position, month of the year, day and time and, finally, atmospheric conditions. 

%It is important to distinguish solar thermal systems into two macrocategories, based on size: large-scale and small-scale systems. The first category can be used in residential districts or industrial operations for domestic hot water, heating support and process heat generation. Large-scale systems are characterized by more complex technologies than small-scale systems, which are generally provided as standardized packages. These systems are mainly used for small-family houses where the energy is gathered in domestic hot water storage tanks. Moreover, large-scale systems need to be adapted to the consumption profile, with a resulting specific planning and dimensioning. The higher technological level makes the large-scale systems more expensive, but this disadvantage is compensated by higher efficiencies and CO2 savings compared to those of small-scale systems.

%According to the vision plan issued by European Solar Thermal Technology Platform (ESTTP), by 2030 up to 50\% of the low and medium temperature heat will be delivered through solar thermal [1].
%However currently, the solar thermal systems are mostly applied to generate hot water in small-scale plants. And when it comes to applications in space heating, large-scale plants in urban heating networks, the insufficient suitable-and-oriented roof of most buildings may dictate solar thermal implementation.
%For a wide market penetration, it is therefore necessary to develop new solar collectors with feasibility to be integrated with building components. Such requirement opens up a large-and-new market segment for the BIST system, especially for district or city-level energy supply in the future.

%BIST is defined as the “multifunctional energy facade” that differs from conventional solar panels in that it offers a wide range of solutions in architectural design features (i.e., colour, texture, and shape), exceptional applicability and safety in construction, as well as additional energy production. It has flexible functions of buildings’ heating/cooling, hot water supply, power generation and simultaneously improvement of the insulation and overall appearance of buildings. This facade based BIST technologies would boost the building energy efficiency and literally turn the envelope into an independent energy plant, creating the possibility of solar-thermal deployment in high-rise buildings.

%Future projections stating the growing gap between energy supply and demand have motivated the development of environmentally benign energy technologies. Among others, solar energy, as a major renewable and eco-friendly energy source with the most prominent characteristic of inexhaustibility, seems to be more promising to offer sustainable solutions towards environmental protection and conservation of conventional energy sources. Thus, solar energy based systems can meet energy demands to some extent to maintain the balance in the ecosystem. However, public acceptance of solar energy technologies depends heavily on factors such as efficiency, cost-effectiveness, reliability and availability [1,2].

%The European Union (EU) has agreed demanding climate and energy targets to be achieved by 2020, known as the “20-20-20” targets. Ireland must implement the EU energy policy objectives for 2020 outlined in the European Strategic Energy Technology Plan: i) a reduction in greenhouse gas emissions of at least 20\% below 1990 levels and 20\% of EU energy consumption replaced by renewable resources, and; ii) a 20\% reduction in primary energy use to be achieved by improving energy efficiency. 
%In the National Renewable Energy Action Plan (NREAP), the government of Ireland has set targets of 40\%, 12.5\% and 10\% of electricity consumption, heating and cooling, and transport, respectively, from renewable energy sources by 2020 (DCENR, 2009).

%Building integrated solar thermal system (BISTS) is a potential solution towards energy efficiency and reduced operational cost in built environment. It is defined as the "multifunctional energy facade" that differs from conventional solar panels in that it offers a wide range of solutions in architectural design features (i.e., colour, texture, and shape), exceptional applicability and safety in construction, as well as additional energy production. For a wide market penetration, it is therefore necessary to develop new solar collectors with feasibility to be integrated with building components. Such requirement opens up a large-and-new market segment for the BIST system, especially for district or city-level energy supply in the future.

%Solar collector installations are usually mounted on roofs rather than forming an integral part of the roof or fa\c{c}ade. The consequences of this non-integration may include aesthetic unacceptance and unaffordable initial installation cost. Full building integration of solar thermal systems has the potential to increase their prevalence by (i) eliminating barriers to their adoption, (ii) improving aesthetic appearance, and (iii) displacing unnecessary roof fa\c{c}ade elements, enhancing economic viability.

%Solar air heaters are devices that convert solar thermal energy into warm air for use in space heating of buildings and dwellings, timber seasoning, and drying of agricultural products (\cite{Alta2010}). Concentrating collectors, which receive solar radiation through an aperture to direct it to a smaller absorber area by reflection, can deliver airflow at temperatures higher than flat-plate collectors (\cite{Duffie2013};\cite{Goswami2015}).

