
%\thispagestyle{empty}
\begin{center}
	\Figure[scale=0.40,placement=!ht,label={logoDIT}]{figs/TUD-logo}
	{\Huge {\bf Building Integration of a Solar Air \\ Heating System}} \\
	\vspace{3cm}
	{\LARGE {\bf Fernando Superbi Guerreiro BSc MSc}} \\
	\vspace*{1.5cm}
	\vfill
	{\large Supervisors: Dr. Michael McKeever, Prof. David Kennedy, \\
	    Prof. Brian Norton \\
		School of Electrical and Electronic Engineering \\
		Technological University Dublin \\
		Republic of Ireland \\
		December 2023 \\
	}
\end{center}

%\newpage
\chapter*{Abstract}
%\thispagestyle{empty}
\vspace*{1cm}

In order to achieve the global carbon emission target, the high fraction of locally available renewable energy sources will become necessary to meet energy demand. Solar energy is one of the most important renewable sources locally available for use in space heating, cooling, hot water supply and power production. Building integrated solar thermal systems (BISTS) can be a potential solution towards the enhanced energy efficiency and reduced operational cost in built environment. The current research aimed at developing an active solar air heating collector for building integration. This system consisted of several asymmetric compound parabolic concentrating collectors with inverted transpired absorber stacked on the top of each other capable of receiving direct solar insolation for 8 hours/day during the summer season. An optical analysis was carried out with the assistance of a 3D ray tracing technique in order to investigate the optical performance through a parametric analysis, and thus to decide the design to be used as a prototype. It has dimensions of 0.42 m x 0.40 m x 1.25 m and concentration ration 2.28, able to absorb up to 67\% of direct solar radiation in the period of operation. Subsequently, the prototype was experimentally tested by using five different air flow rates (22 -- 60 kg/h) in open loop configuration. Experimental results show that, on clear days and in periods of high solar insolation (greater than 800 W/m$^{\rm{2}}$), the airflow temperature rise varied between 12 $\rm{^oC}$ (at the highest airflow rate) and 27 $\rm{^oC}$ (at the smallest rate), characterising thus thermal efficiencies between 52 -- 62\%. After experimental characterisation, a thermal modelling based on energy balance was undertaken with the objective of simulating the collector’s thermal performance for further control of multiple units connected altogether. Validation of this model indicates that it predicts data of air temperature delivered and energy collected with uncertainties of $\pm$ 2 $\rm{^oC}$ and $\pm$ 50 W/m$^{\rm{^2}}$ of absorber area, respectively. The thermal modelling was also used to predict the thermal performance of three collectors connected in series and in parallel. The connection in series was able to deliver higher outlet airflow temperatures (higher than 70 $\rm{^oC}$ on a clear day), whereas the connection in parallel can collect more useful heat but delivering the same airflow temperature as if it is leaving the first collector in series. Lastly, this model was used for simulating barley drying. Three scenarios were employed to fulfill energy requirements: using only a gas burner to heat the airflow to 60°C, combining a solar heating system with a gas burner to achieve the same temperature, and solely relying on the solar heating system without an air temperature limit. For a specific airflow level, the gas burner system consumed the most gas and had a solar fraction of 0, indicating full reliance on gas for heating. Conversely, the standalone Solar Air Heating System (SAHS) yielded less dried mass, but both configurations achieved a solar fraction of 1, signifying exclusive reliance on solar energy. Results demonstrate that incorporating solar air heating systems can significantly reduce the need for natural gas heating. Specifically, the gas burner + SAHS in parallel exhibited the lowest gas consumption and the highest solar fraction among all the gas burner-based systems.

 %Lastly, the collector was evaluated by a CFD analysis to have a better understanding of heat transfer within the collector. The results show that the use of the forced air flowing through the collector’s cavity enhances the convective losses suppression of the system.

%\newpage
%\relax
\chapter*{Declaration}
%\thispagestyle{empty}
\vspace*{1cm}

I certify that this thesis which I now submit for examination for the award of PhD, is entirely my own work and has not been taken from the work of others, save and to the extent that such work has been cited and acknowledged within the text of my work.

This thesis was prepared according to the regulations for graduate study by research of the Technological University Dublin and has not been submitted in whole or in part for another award in any other third level institution.

The work reported on in this thesis conforms to the principles and requirements of the TU Dublin's guidelines for ethics in research.


Signature

Date

Candidate

%\newpage
\chapter*{Acknowledgement}
%\thispagestyle{empty}
\vspace*{1cm}

I would like to express my sincere appreciation to the individuals and organizations who have provided support and guidance throughout my PhD journey. I am deeply grateful to my supervisors, Professor Mick McKeever, Professor David Kennedy, and Professor Brian Norton, for their unwavering encouragement, support, and valuable insights.

In addition, I would like to thank my thesis committee members for their constructive criticism and feedback, which significantly improved the quality of my research.

I am thankful for the welcoming and supportive research environment created by the staff and students of TU Dublin. Working with inspiring and talented individuals has been a privilege, and I am grateful for the collaborations and friendships that have emerged from this community.

My family has been a constant source of love, support, and encouragement throughout my academic journey, and I am deeply indebted to them for their sacrifices and unwavering belief in my abilities.

I would also like to acknowledge the invaluable support and encouragement of my friends, who have provided inspiration and motivation during the ups and downs of my research journey.

Finally, I would like to acknowledge the National Council for Scientific and Technological Development (CNPq: process number 233789/2014-6) for providing me financial support to complete this PhD.

