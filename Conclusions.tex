\chapter{Conclusions}
\label{Cap:Con}
%

%The conclusions from this PhD work are listed as follows:
%
%\begin{itemize}
%	\item Development of a 3D ray tracing for optical analysis of direct radiation to assist the selection of a collector design: an optical analysis has been undertaken to evaluate its optical efficiency considering factors as: glazing transmittance, truncation level, length and tertiary section height. The results show that there is a maximum value of optical efficiency for a particular parabolic reflector shape and a maximum value of glazing transmittance at different inclinations. The proposed air heater concentrator is able to absorb in average 67\% of direct solar radiation during most part of the day in the period evaluated;
%	
%	\item Performance of experimental work to characterise the specific type of collector:
%	the experiments were carried out in open loop configuration, where the air was blown by a 12-W fan with a voltage adaptor to vary the airflow rate. Experimental results were analysed for different airflow rates ranging between 0.04 and 0.115 kg/m$^2$.s. Results show that the maximum outlet air temperatures at solar noon (13:00 to 14:00) varied from 40 $^{\rm{o}}$C at the highest airflow to 52 $^{\rm{o}}$C at the lowest and the thermal efficiency varied from 52\% to 62\%. 
%	
%	\item Thermal modelling of the solar air heating concentrator and validation against experimental results: it was shown that, in general, the heat transfer model underestimates 65\% of the data and 95\% of the residues generated are between $\pm$2 $^{\rm{o}}$C and $\pm$50 W/m$^2$. In overall, the mean absolute error in terms of temperature was found to be 2\% and in terms of useful energy was 8\%. The model was also used to simulate results at nearly steady state condition. The thermal efficiency curve was plotted and compare it to the experimental one. The parameters of the Hottel-Whillier-Bliss equation were calculated by the thermal model with relative difference of 2\% and 10\%, respectively, in relation to the parameters of the experimental data.
%	
%	\item Thermal simulation of a system consisted of more than one collector connected together: The thermal modelling was also used to predict the thermal performance of three connected collectors in series and in parallel. The connection in series is able to deliver higher outlet airflow temperatures whereas the connection in parallel can collect more useful heat but delivering the same airflow temperature as if it is leaving the first collector in series.
%	
%	
%	\item Use of this thermal model for simulating barley drying: Three scenarios were considered to meet the energy demand: simulating only a gas burner to heat the airflow up to 60 $^{\rm{o}}$C; simulating the solar heating system and a gas burner to heat the airflow up to 60 $^{\rm{o}}$C, and; simulating only the solar heating system to heat the airflow with no air temperature limit. From an specific airflow level used in the simulations, the gas burner system has the highest gas volume consumption and a solar fraction of 0, meaning it relies completely on gas for heating. On the other hand, the SAHS standalone system produces less dried mass, but both connections have a solar fraction of 1, indicating that they rely solely on solar energy.
%	
%	
%\end{itemize}

%\begin{itemize}
%	\item Development of a 3D ray tracing method for optical analysis of direct radiation to assist in the selection of a collector design: An optical analysis was conducted to evaluate its optical efficiency, considering factors such as glazing transmittance, truncation level, length, and tertiary section height. The results indicate a maximum optical efficiency for a specific parabolic reflector shape and a peak glazing transmittance at different inclinations. The proposed air heater concentrator demonstrated the ability to absorb an average of 67\% of direct solar radiation during the evaluated period of the day.
%	
%	\item Performance of experimental work to characterise the specific type of collector: Experiments were carried out in an open-loop configuration, where air was blown using a 12-W fan with a voltage adaptor to vary the airflow rate. Experimental results were analysed for different airflow rates ranging from 0.04 to 0.115 kg/m$^2\cdot$s. The findings reveal that the maximum outlet air temperatures at solar noon (13:00 to 14:00) varied from 40\,$^{\circ}$C at the highest airflow to 52\,$^{\circ}$C at the lowest, with thermal efficiencies ranging between 52\% and 62\%.
%	
%	\item Thermal modelling of the solar air heating concentrator and validation against experimental results: The heat transfer model was found to underestimate 65\% of the data, with 95\% of the residuals falling within \(\pm2\,^{\circ}\text{C}\) and \(\pm50\,\text{W/m}^2\). Overall, the mean absolute error in temperature was 2\%, and in useful energy, it was 8\%. The model was also used to simulate results under nearly steady-state conditions. The thermal efficiency curve was plotted and compared with experimental data. The parameters of the Hottel-Whillier-Bliss equation were calculated by the thermal model, showing relative differences of 2\% and 10\%, respectively, compared to the experimental parameters.
%	
%	\item Thermal simulation of a system comprising multiple collectors connected together: The thermal model was employed to predict the thermal performance of three collectors connected in series and in parallel. The series connection was found to deliver higher outlet airflow temperatures, while the parallel connection captured more useful heat but delivered airflow temperatures equivalent to those from the first collector in the series.
%	
%	\item Application of the thermal model for simulating barley drying: Three scenarios were analysed to meet the energy demand: (1) simulating only a gas burner to heat airflow to 60\,$^{\circ}$C; (2) simulating the solar heating system combined with a gas burner to achieve the same temperature; and (3) simulating the solar heating system alone, without a temperature limit. At a specific airflow level, the gas burner system exhibited the highest gas consumption and a solar fraction of 0, relying entirely on gas for heating. Conversely, the standalone solar air heating system produced less dried mass, but both configurations had a solar fraction of 1, indicating complete reliance on solar energy.

% \end{itemize}

%This chapter has consolidated the findings and highlighted key milestones achieved during the study of the proposed solar air heater concentrator and its applications.
%
%The design of an air heater concentrator with a horizontally downward-facing absorber proved effective in concentrating solar thermal energy and minimising heat losses. Through 3D ray-tracing optical analysis, the study identified an optimal parabolic reflector shape and glazing transmittance for different inclinations. A tertiary section was found to enhance uniformity in energy distribution across the absorber, although it introduced higher optical losses. Further exploration of the tertiary section's impact through thermal modelling and outdoor experiments remains necessary. The developed concentrator demonstrated an average absorption of 67\% of direct solar radiation during the evaluation period, showcasing its potential efficiency.
%
%The fabrication of a solar air heating prototype was successfully detailed, with materials selected based on their suitability for solar energy applications. The carbon fibre weave with 85\% absorptivity was chosen as the absorber, while the \textit{Mirosun} reflector sheet with 95\% reflectivity and tempered clear glass with 90\% transmittance served as the reflector and glazing, respectively. The experimental setup included a 12-W fan for airflow adjustment, enabling tests at varying flow rates. Results indicated that outlet air temperatures ranged from 40$^{\circ}$C  at the highest airflow to 52$^{\circ}$C  at the lowest, while thermal efficiency varied between 52\% and 62\% under peak solar noon conditions.
%
%Experimental findings were complemented by thermal modelling and simulation. Validation of the thermal model against experimental data showed a mean absolute error of 2\% for temperature and 8\% for useful energy. The parameters of the Hottel-Whillier-Bliss equation derived from the experimental data ( and ) closely matched those obtained from the model, with relative differences of 2\% and 10\%, respectively. The model’s robustness enabled simulations to predict performance under steady-state conditions and explore configurations of connected collectors. Collectors in series provided higher outlet air temperatures, whereas parallel connections increased useful heat collection without raising airflow temperatures beyond the initial collector’s output.
%
%The system’s potential for practical applications, such as barley drying, was investigated using the thermal model. Three scenarios highlighted the role of solar air heating systems (SAHS) in reducing dependence on natural gas. The standalone SAHS demonstrated a solar fraction of 1, while the gas burner + SAHS parallel configuration achieved the lowest gas consumption and highest solar fraction among hybrid systems. These findings underscore the potential for significant reductions in natural gas reliance, contingent on sunlight availability, natural gas costs, and system accessibility.
%
%In summary, this study has successfully designed, fabricated, and evaluated a solar air heating system. Key milestones include identifying optimal design parameters, validating a thermal model against experimental data, and demonstrating the system’s practical applications. The results reinforce the viability of solar air heating as a sustainable and efficient alternative to conventional energy sources, with opportunities for further optimisation and scalability.

This chapter concludes the research undertaken for this PhD by summarising its key findings, revisiting the objectives and contributions, and outlining potential directions for future work. The study focused on the development, testing, and modelling of a solar air-heating concentrator, aiming to enhance its design and performance for building-integrated applications. In the final phase, the system’s modelling was applied to address the energy demands of a barley drying operation. 

The chapter is organized as follows:

\begin{itemize}
	\item A detailed summary of the primary conclusions of this research and key findings, addressing the specific objectives described in Chapter \ref{Cap:Int};
	\item Suggestions for future research to build upon the outcomes.
\end{itemize}


\section{Main conclusions and key findings}

The main conclusions drawn from this PhD research are summarised as follows:

\begin{itemize}
	\item \textbf{Development of a 3D ray-tracing model for optical analysis of direct radiation to assist in the selection of a collector design}: \\
	A comprehensive optical analysis was conducted to evaluate the optical efficiency of the proposed collector design, considering factors such as glazing transmittance, truncation level, length, and tertiary section height. The findings reveal that there was an optimal parabolic reflector shape that maximises optical efficiency, as well as a peak glazing transmittance value for specific inclinations. The analysis was also able to calculate the effect of the tertiary section height on the energy distribution over the absorber plate. All of those design parameters were evaluated to select the concentrator's shape and size for further fabrication: concentration ratio of 2.28 and capability to absorb, on average, 67\% of direct solar radiation during the evaluated period for most of the day;
	
	\item \textbf{Performance of experimental work to characterise the specific type of collector}: \\
	Experiments were conducted in an open-loop configuration, where air was blown by a 12-W fan with a voltage adaptor used to vary the airflow rate. The experimental data were obtained for airflow rates ranging from 0.04 to 0.115 kg/(m$^2\cdot$s). The results indicate that the maximum outlet air temperatures at solar noon (13:00 to 14:00) ranged from 40 $^{\circ}$C at the highest airflow rate to 52 $^{\circ}$C at the lowest. The thermal efficiency varied between 52\% and 62\%, depending on the airflow rate;
	
	\item \textbf{Thermal modelling of the solar air heating concentrator and validation against experimental results}: \\
	A heat transfer model was developed and validated against experimental data. Although the model underestimated 65\% of the experimental values, 95\% of the residuals were within $\pm$2~$^{\circ}$C for temperature and $\pm$50~W/m$^2$ for useful energy. The mean absolute error was determined to be 2\% for temperature and 8\% for useful energy. Additionally, the model was employed to simulate steady-state conditions, with the resulting thermal efficiency curve compared to experimental data. The parameters of the Hottel-Whillier-Bliss equation were derived from the thermal model, yielding relative differences of 2\% and 10\%, respectively, when compared to the experimentally obtained parameters.
	
	\item \textbf{Thermal simulation of a system comprising multiple collectors connected together}: \\
	The thermal model was extended to predict the performance of systems with three collectors connected in series and parallel configurations. The series configuration yielded higher outlet air temperatures, while the parallel configuration collected more useful heat but delivered airflow temperatures equivalent to those exiting the first collector in the series setup;
	\\
	
	\item \textbf{Application of the thermal model for simulating barley drying}: \\
	Three scenarios were evaluated to meet the energy demand for barley drying: 
	\begin{enumerate}
		\item Simulating a gas burner to heat the airflow up to 60~$^{\circ}$C.
		\item Simulating a hybrid system with a solar heating system and a gas burner to heat the airflow up to 60~$^{\circ}$C.
		\item Simulating the solar heating system alone, with no upper temperature limit for the airflow.
	\end{enumerate}
	The simulations revealed that the gas burner system had the highest gas consumption, with a solar fraction of 0, indicating total reliance on gas for heating. Conversely, the standalone solar air heating system (SAHS) produced less dried mass but achieved a solar fraction of 1 in both series and parallel configurations, demonstrating complete reliance on solar energy. Additionally, the results indicate that the combination of gas burner + SAHS in parallel configuration exhibited the lowest gas consumption and highest solar fraction among all the listed systems with gas burner.
	
\end{itemize}
	


%The suggestions for the future are listed as follows:
%
%\begin{itemize}
%	\item Use the ray tracing technique to simulate diffuse radiation in 3D to have better results for optical modelling;
%    \item Run experiments using more than one solar air heating collector connected to find the best configuration in terms of air temperature delivered and energy collection.
%    \item Perform a Computational Fluid Dynamics (CFD) modelling to better understand the heat transfer within the concentrator. Simulations will be undertaken to investigate the effect of the tertiary section height and the collector's size;
%    \item Develop a control algorithm to deliver airflow temperature at a certain set-point and further use of proper equipment to maintain this air temperature at the set-point according to the application required.
%    \item Couple a solar air heating system to a real dryer and run experiments to measure how much barley can be dried at a fixed air temperature.
%\end{itemize}

\section{Suggestions for future work}

From the findings of this research, several areas are identified for further exploration:

\begin{itemize}
	\item \textbf{Use of the ray tracing technique to simulate diffuse radiation in 3D for improved optical modelling}: \\
	Extending the ray tracing methodology to include simulations of diffuse radiation will enhance the accuracy of optical modelling. This improvement is expected to provide a more comprehensive understanding of the concentrator's performance under realistic solar radiation conditions.
	
	\item \textbf{Conducting experiments with multiple solar air heating collectors connected together}: \\
	Experimental studies should investigate the performance of systems comprising multiple collectors connected in series and parallel configurations. These experiments will aim to identify the optimal configuration in terms of delivered air temperature and overall energy collection efficiency.
	
	\item \textbf{Performing Computational Fluid Dynamics (CFD) modelling to analyse heat transfer within the concentrator}: \\
	Advanced CFD simulations should be undertaken to gain deeper insights into the heat transfer mechanisms within the concentrator. These simulations can explore the effects of design parameters such as the tertiary section height and collector size on thermal performance, thereby informing future design improvements.
	
	\item \textbf{Developing a control algorithm for maintaining a set-point airflow temperature}: \\
	A control algorithm should be devised to regulate the airflow temperature at a desired set-point, tailored to the specific application. Additionally, appropriate equipment, such as thermostats or variable-speed fans, should be integrated to ensure that the system maintains the set-point temperature efficiently under varying operating conditions.
	
	\item \textbf{Coupling the solar air heating system with a real dryer for experimental evaluation}: \\
	A solar air heating system should be coupled with a functional drying system to perform experiments on real-world applications, such as barley drying. These experiments should aim to quantify the amount of barley that can be dried at a controlled air temperature, providing valuable data for assessing the system's feasibility and efficiency in agricultural or industrial settings.
\end{itemize}




