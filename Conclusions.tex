\chapter{Conclusions}
\label{Cap:Con}
%

The conclusions from this PhD work are listed as follows:

\begin{itemize}
	\item Development of a 3D ray tracing for optical analysis of direct radiation to assist the selection of a collector design: an optical analysis has been undertaken to evaluate its optical efficiency considering factors as: glazing transmittance, truncation level, length and tertiary section height. The results show that there is a maximum value of optical efficiency for a particular parabolic reflector shape and a maximum value of glazing transmittance at different inclinations. The proposed air heater concentrator is able to absorb in average 67\% of direct solar radiation during most part of the day in the period evaluated;
	
	\item Performance of experimental work to characterise the specific type of collector:
	the experiments were carried out in open loop configuration, where the air was blown by a 12-W fan with a voltage adaptor to vary the airflow rate. Experimental results were analysed for different airflow rates ranging between 0.04 and 0.115 kg/m$^2$.s. Results show that the maximum outlet air temperatures at solar noon (13:00 to 14:00) varied from 40 $^{\rm{o}}$C at the highest airflow to 52 $^{\rm{o}}$C at the lowest and the thermal efficiency varied from 52\% to 62\%. 
	
	\item Thermal modelling of the solar air heating concentrator and validation against experimental results: it was shown that, in general, the heat transfer model underestimates 65\% of the data and 95\% of the residues generated are between $\pm$2 $^{\rm{o}}$C and $\pm$50 W/m$^2$. In overall, the mean absolute error in terms of temperature was found to be 2\% and in terms of useful energy was 8\%. The model was also used to simulate results at nearly steady state condition. The thermal efficiency curve was plotted and compare it to the experimental one. The parameters of the Hottel-Whillier-Bliss equation were calculated by the thermal model with relative difference of 2\% and 10\%, respectively, in relation to the parameters of the experimental data.
	
	\item Thermal simulation of a system consisted of more than one collector connected together: The thermal modelling was also used to predict the thermal performance of three connected collectors in series and in parallel. The connection in series is able to deliver higher outlet airflow temperatures whereas the connection in parallel can collect more useful heat but delivering the same airflow temperature as if it is leaving the first collector in series.
	
	
	\item Use of this thermal model for simulating barley drying: Three scenarios were considered to meet the energy demand: simulating only a gas burner to heat the airflow up to 60 $^{\rm{o}}$C; simulating the solar heating system and a gas burner to heat the airflow up to 60 $^{\rm{o}}$C, and; simulating only the solar heating system to heat the airflow with no air temperature limit. From an specific airflow level used in the simulations, the gas burner system has the highest gas volume consumption and a solar fraction of 0, meaning it relies completely on gas for heating. On the other hand, the SAHS standalone system produces less dried mass, but both connections have a solar fraction of 1, indicating that they rely solely on solar energy.
	
	
\end{itemize}

The suggestions for the future are listed as follows:

\begin{itemize}
	\item Use the ray tracing technique to simulate diffuse radiation in 3D to have better results for optical modelling;
    \item Run experiments using more than one solar air heating collector connected to find the best configuration in terms of air temperature delivered and energy collection.
    \item Perform a Computational Fluid Dynamics (CFD) modelling to better understand the heat transfer within the concentrator. Simulations will be undertaken to investigate the effect of the tertiary section height and the collector's size;
    \item Develop a control algorithm to deliver airflow temperature at a certain set-point and further use of proper equipment to maintain this air temperature at the set-point according to the application required.
    \item Couple a solar air heating system to a real dryer and run experiments to measure how much barley can be dried at a fixed air temperature.
\end{itemize}







